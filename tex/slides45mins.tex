\documentclass[12pt,english, aspectratio=169]{beamer}



% my default preamble

\usepackage{amsmath}
\usepackage{amssymb}
\usepackage{bm}
\usepackage{mathpazo}
% font setup
\usepackage[sfdefault,lf]{carlito}
\usefonttheme[onlymath]{serif}
\usepackage{xcolor}
\usepackage{tcolorbox}
\usepackage{hyperref}
\usepackage[round]{natbib}
\usepackage{adjustbox}
\usepackage{multicol}
\usepackage{array}
\usepackage{booktabs}
\usepackage{dcolumn}
\usepackage{versions}
\usepackage{xifthen}
\usepackage[normalem]{ulem}
\usepackage{tikz}
\usetikzlibrary{positioning}
\usetikzlibrary{snakes}
\usetikzlibrary{calc}
\usetikzlibrary{decorations.markings}
\usetikzlibrary{shapes.misc}
\usetikzlibrary{matrix,shapes,arrows,fit,tikzmark}
\renewcommand{\arraystretch}{1.2} 



\newcounter{saveenumi}
\newcommand{\seti}{\setcounter{saveenumi}{\value{enumi}}}
\newcommand{\conti}{\setcounter{enumi}{\value{saveenumi}}}

% define long or short presentation
% short: 45 mins
% long: 75-90 mins
\newboolean{longv}
\newboolean{shortv}
% \setboolean{shortv}{true}

% define the path to dropbox
\newcommand{\Dropbox}{${HOME}/Dropbox/research/mobility/output}
\newcommand{\DropboxRoot}{${HOME}/Dropbox}

% \newcommand{\newblock}{}

% \usepackage{enumitem}

% \usepackage{heuristica}
% \usepackage[heuristica,vvarbb,bigdelims]{newtxmath}
% \usepackage[T1]{fontenc}
% \renewcommand*\oldstylenums[1]{\textosf{#1}}

% \usepackage{lmodern}
% \usepackage[T1]{fontenc}

% \usepackage[bitstream-charter]{mathdesign}
% \usepackage[T1]{fontenc}

% \usepackage{cmbright}
% \usepackage[T1]{fontenc}
% \usetheme{Singapore}
% \setbeamertemplate{frametitle}[default][center]

% set some beamer templates
\setbeamertemplate{footline}[frame number] 
% \setbeamertemplate{caption}[numbered]
% \setbeamertemplate{footline}[frame number]{}
% \setbeamertemplate{navigation symbols}{}
% \setbeamertemplate{footline}{}
\setbeamertemplate{frametitle}[default][center]

% define new commands
\newcommand{\red}[1]{{\color{red}{#1}}}
\newcommand{\blue}[1]{{\color{blue}{#1}}}
\newcommand{\bred}[1]{\textbf{\color{red}{#1}}}
\newcommand{\bblue}[1]{\textbf{\color{blue}{#1}}}
\definecolor{green}{RGB}{0,158,115}
\newcommand{\bgreen}[1]{\textbf{\color{green}{#1}}}

\newtcbox{\cyanbox}{on line, arc=1pt,left=1pt,right=1pt,top=1pt,bottom=1pt, colback=cyan!35!white, colframe=cyan!75!black,boxrule=1pt}
\newtcbox{\bluebox}{on line, arc=1pt,left=1pt,right=1pt,top=1pt,bottom=1pt, colback=blue!15!white, colframe=blue!80!black,boxrule=1pt}
\newtcbox{\redbox}{on line, arc=1pt,left=1pt,right=1pt,top=1pt,bottom=1pt, colback=red!5!white, colframe=red!75!black,boxrule=1pt}
\newtcbox{\greybox}{on line, arc=1pt,left=0.2pt,right=0.2pt,top=0.2pt,bottom=0.2pt, colback=black!5!white, colframe=white!75!black,boxrule=0.5pt}
\newcommand{\link}[2]{\greybox{\hyperlink{#1}{\texttt{#2}}}}
\newcommand{\slink}[2]{\greybox{\hyperlink{#1}{{\small\texttt{#2}}}}}

% \renewcommand<>{\item}{\beameroriginal\item\vspace{\stretch{.25}}}

% new variable linewidth environments
\newenvironment{squarei}
{  \setbeamertemplate{itemize items}[square]
  \begin{itemize}    }
{ \end{itemize}                  } 
\newenvironment{smalli}
{ \begin{itemize}
    \setlength{\itemsep}{1pt}
    \setlength{\parskip}{1pt}
    \setlength{\parsep}{1pt}     }
{ \end{itemize}                  } 
\newenvironment{widei}
{ \begin{itemize}
    \setlength{\itemsep}{10pt}
    \setlength{\parskip}{10pt}
    \setlength{\parsep}{10pt}     }
{ \end{itemize}                  } 
\newenvironment{smalle}
{ \begin{enumerate}
    \setlength{\itemsep}{1pt}
    \setlength{\parskip}{1pt}
    \setlength{\parsep}{1pt}     }
{ \end{enumerate}                  } 
\newenvironment{widee}
{ \begin{enumerate}
    \setlength{\itemsep}{10pt}
    \setlength{\parskip}{10pt}
    \setlength{\parsep}{10pt}     }
{ \end{enumerate}                  } 
\newenvironment{mide}
{ \begin{enumerate}
    \setlength{\itemsep}{5pt}
    \setlength{\parskip}{5pt}
    \setlength{\parsep}{5pt}     }
{ \end{enumerate}                  } 
\newenvironment{midi}
{ \begin{itemize}
    \setlength{\itemsep}{5pt}
    \setlength{\parskip}{5pt}
    \setlength{\parsep}{5pt}     }
{ \end{itemize}                  } 
% different enumerates
\setbeamercolor{item projected}{bg=blue!80!black,fg=white}
\setbeamertemplate{enumerate items}[circle]
\setbeamertemplate{itemize items}[circle]
\setbeamertemplate{itemize subitems}[triangle]


% transition frame
\newenvironment{transitionframe}{
  \setbeamercolor{background canvas}{bg=yellow}
  \begin{frame}}{
    \end{frame}
}



%%% TIKZ STUFF
\tikzset{   
        every picture/.style={remember picture,baseline},
        every node/.style={anchor=base,align=center,outer sep=1.5pt},
        every path/.style={thick},
        }
\newcommand\marktopleft[1]{%
    \tikz[overlay,remember picture] 
        \node (marker-#1-a) at (-.3em,.3em) {};%
}
\newcommand\markbottomright[1]{%
    \tikz[overlay,remember picture] 
        \node (marker-#1-b) at (.1em,.3em) {};%
    % \tikz[overlay,remember picture,inner sep=3pt]
        % \node[draw=red,rounded corners,fit=(marker-#1-a.north west) (marker-#1-b.south east)] {};%
}
\tikzstyle{every picture}+=[remember picture] 
\tikzstyle{mybox} =[draw=black, very thick, rectangle, inner sep=10pt, inner ysep=20pt]
\tikzstyle{fancytitle} =[draw=black,fill=red, text=white]
%%%% END TIKZ STUFF


% set length of presentation
\setboolean{longv}{false}
\ifthenelse{\boolean{longv}}{%
  \includeversion{longv}
  % \excludeversion{shortv}
}{%
  \excludeversion{longv}
  % \includeversion{shortv}
}

\author{Florian Oswald, SciencesPo}
\date{OECD 2019}

\begin{document}


\title{
The Effect of Homeownership on the Option Value of Regional Migration
}

\frame{\titlepage}

\begin{frame}{How Important Is Migration?}
\begin{midi}
\item US Regional migration rates are relatively low.
\item \onslide+<2->Does that mean migration does not matter?
\item \onslide+<3->\bred{Question}: What is the option value of being able to migrate, and how does this interact with homeownership?
\item \onslide+<4->Optimal housing and labor market policies depend on
this.
\end{midi}
\end{frame}

\begin{frame}{Who Is Likely To Move?}{And Why Does it Matter?}

\begin{midi}
 \item \redbox{Renters} move much more than \bluebox{owners}.
 \item \onslide+<2->Ownership status is endogenous.
 	\begin{itemize}
		 \item \onslide+<3->$\Rightarrow$ Model housing and migration jointly!
		 \item \onslide+<4->$\Rightarrow$ Over the lifecycle!
		 \item \onslide+<5->$\Rightarrow$ With relatistic uncertainty in prices and mortgage financing!
 	\end{itemize}
 	\item \onslide+<6->Heterogeneous migration elasticites, heterogenous WTP for migration.
 \end{midi} 

\end{frame}

\begin{frame}{Outline}

In this paper, I build a lifecycle model that allows for

\begin{itemize}
\item Regional risk in house prices and labor income,
\item Migration, 
\item Saving in assets,
\item The choice to buy or rent a home.
\end{itemize}
\medskip{}

\begin{itemize}
\item \onslide+<2->Estimate the preference parameters of the model using
both micro and macro data.
\item \onslide+<2->Develop a measure of the option value of migration.

\end{itemize}
\end{frame}

\begin{frame}{Contributions}

\begin{widee}
\item Adds to \cite{kennan2003effect}: 

\begin{itemize}
\onslide+<2->\item regional shocks,
\onslide+<2->\item assets and 
\onslide+<2->\item housing choices.
\end{itemize} 
\item \onslide+<3->Takes regional house price and income risk seriously (Shocks are not IID!).
\item \onslide+<4->Provides a fully structural implementation of \cite{yagan,yagan2017employment}: \emph{Migratory Insurance}. 
\end{widee}
\end{frame}

% \begin{frame}{Roadmap}

% \tableofcontents{}
% \end{frame}

\section{Literature}
\begin{frame}{Related literature}

\begin{widei}
\item Dynamic models of migration over lifecycle: {\footnotesize{}\cite{baum2012understanding}, \cite{kennan2003effect}, \cite{gemici2007family}, \cite{ransom2018labor}\vspace{3mm}}
\item Static spatial equilibrium models of migration, and local labor markets:
{\footnotesize{}\cite{notowidigdo2011incidence}, \cite{Moretti:2011fk}, \cite{diamond2016determinants},
\cite{piypromdee}, \cite{monras2018economic}\vspace{3mm}}
\item Regional Fluctuations and Unemployment: {\footnotesize{}\cite{blanchard1992regional}, \cite{schmutz}}
\item  \cite{yagan,yagan2017employment}: Moving to Opportunity?
\end{widei}
\end{frame}



\begin{longv}
\begin{frame}{Related literature }

\begin{itemize}
\item On the effects of 2007 housing bust on unemployment and mobility,
and mismatch: 

\begin{itemize}
\item {\footnotesize{}empirical: Ferreira et al. (2010), Schulhofer-Wohl
(2011), Demyanyk et al. (2013)}{\footnotesize\par}
\item {\footnotesize{}theoretical: Shimer (2005), Nenov (2012), Song et
al. (2014) and Karahan and Rhee (2011)\vspace{3mm}}{\footnotesize\par}
\end{itemize}
\item Housing and Labor reallocation: 

\begin{itemize}
\item {\footnotesize{}Davis, Fisher and Veracierto (2010, 2013)}{\footnotesize\par}
\item {\footnotesize{}Wasmer (2009)\vspace{3mm}}{\footnotesize\par}
\end{itemize}
\item Housing demand over the lifecycle and mortgage default:{\footnotesize{}
Li, Meghir and Oswald (2014)\vspace{3mm}}{\footnotesize\par}

{\footnotesize{}}{\footnotesize\par}
\end{itemize}
\end{frame}
\end{longv}
\section{Empirical Facts}

\begin{frame}[plain]{}

\vspace{1cm}

\begin{center}
\textbf{\textcolor{blue}{\LARGE{}Facts}}{\LARGE\par}
\par\end{center}

\end{frame}

\begin{frame}{Definition of \emph{Region}: Census Division\hypertarget{Divisions}{}}

\begin{widei}
\item I assume regions correspond to Census Divisions (9 in total).
\item Each region is a market with uniform house price and average level
of labor productivity, but individual specific shocks to income.
\item Moves within a region are not counted as migration.
\item 70\% of interstate moves are cross Division moves.

\end{widei}
%
\end{frame}
\begin{frame}[plain]{}
\begin{adjustbox}{center}
\includegraphics[height=1.1\paperheight]{\Dropbox/maps/divisions}\end{adjustbox}
\end{frame}

% \begin{frame}[plain]{}

% \includegraphics[height=0.95\paperheight]{\Dropbox/data/sipp/agg_reg_p}
% \end{frame}

% \begin{frame}[plain]{}

% \includegraphics[height=0.95\paperheight]{\Dropbox/data/sipp/agg_reg_y}
% \end{frame}


\begin{frame}{Annual Migration Rates}

\vspace{1cm}
\begin{table}
\begin{centering}
\input{\Dropbox/data/Molloy/molloy.tex}
\par\end{centering}
\caption{\cite{molloy2011internal}}
\end{table}
\end{frame}

\begin{frame}{Reasons for Moving: CPS Data}

\begin{table}[p]
\begin{centering}
\input{\Dropbox/data/CPS/main-reason-pres1.tex}
\par\end{centering}
\caption{CPS 2013 variable \texttt{NX1RES}: \emph{What was your main reason for moving to this house (apartment)?} Median distance moved is 1347 KM. }
\end{table}

\end{frame}

% \begin{frame}{Reasons for Moving: CPS Data}

% \begin{table}[p]
% \begin{centering}
% \begin{center}
% \input{\Dropbox/data/CPS/main-reason-presKM.tex}
% \par\end{center}
% \par\end{centering}
% \begin{centering}
% \par\end{centering}
% \caption{CPS 2013 data on main motivation of moving, conditional on a cross--Division move. Median distance moved is 1347 KM. }
% \end{table}

% \end{frame}

\begin{frame}{Main Predictors of Migration: Ownership and College}
\begin{equation*}	
\Pr\left(\text{Move across Divisions}|x_{it}\right) = \Phi\left(x_{it}\beta\right)
\end{equation*}	

\uncover<2->{
\begin{table}
\begin{centering}
\input{\Dropbox/data/sipp/D2D-probit-edit-slides.tex}\end{centering}
\caption{SIPP. Additional Controls: $\text{Age}^2$, Wealth, Price/Income.}
\end{table}
}
\uncover<3->{\tikz[overlay,remember picture,inner sep=1pt]
\node[draw=blue,rounded corners,fit=(marker-a1-a.north west) (marker-a1-b.south east)]{};
\tikz[overlay,remember picture,inner sep=1pt]
\node[draw=green,rounded corners,fit=(marker-a2-a.north west) (marker-a2-b.south east)]{};
}
\end{frame}


\begin{frame}{Migration Rates in \bgreen{Non-College Sample}}

\begin{table}
\input{\Dropbox/data/sipp/move_rates.tex}
\caption{SIPP subset to non-college degree household heads.}
\end{table}

\end{frame}


\begin{frame}[plain]{}

\begin{adjustbox}{center}
\includegraphics[height=1\paperheight]{\Dropbox/data/sipp/raw-moversD2D_v2}
\end{adjustbox}
\end{frame}


\begin{frame}{Regional Prices are \textbf{not} IID}

\begin{adjustbox}{center}
\includegraphics[width=1\linewidth]{\Dropbox/data/FHFA/detrended.pdf}\end{adjustbox}

\end{frame}


\begin{frame}{Summary of Facts}


\begin{widee}
\item Main Reasons to Move: Work, Housing and Family
\item \onslide+<2->Homeownership and College are important predictors. 
	\begin{itemize}
	\item $\Rightarrow$ Subset to \bgreen{non-college}!
	\end{itemize}
\item \onslide+<3->Unconditional migration cross Division migration rate: 0.99\%.
\item \onslide+<4->\red{Renters} move at twice the rate of \blue{owners}, at all ages.
\item \onslide+<5->Regional income and house price risk is \emph{not} IID.
\end{widee}

\end{frame}

\section{Model}
\begin{frame}[plain]{}

\vspace{1cm}

\begin{center}
\textbf{\textcolor{blue}{\LARGE{}Model}}{\LARGE\par}
\par\end{center}

\end{frame}
\begin{frame}{Decision Process}

In each period, household $i$ decides 
\begin{midi}
\item How much to save,
\item Whether to own or rent, and
\item Where to live.
\end{midi}
\vspace{0.5cm}
These choices are subject to
\begin{midi}
\item Budget constraints, and
\item Beliefs about the future.
\end{midi}
\end{frame}

\begin{frame}{Decision Process}

In each period, household $i$ decides 
\begin{midi}
\item How much to save: $a_{it+1}$, and
\item Whether to own or rent: $h_{it}\in\{0,1\}$, and
\item Where to live: choose region $d\in D$, with house price $p_{dt}$
and labor productivity index $q_{dt}$.
\end{midi}
\vspace{0.5cm}
These choices are subject to
\begin{midi}
\item Budget constraints: $\left(a_{it},z_{it},h_{it-1},q_{dt},p_{dt}\right)$.
\item Beliefs about the future: $\left(p_{dt+1},q_{dt+1},z_{it+1},s_{it+1}\right)$.
\end{midi}
\end{frame}

\begin{frame}{Location Choice Problem}

\begin{eqnarray*}
V_{t}\left(x_{it}\right) & = & \max_{d'\in D}\left\{ v_{t}\left(x_{it},d'\right)+\varepsilon_{id't}\right\} \\
\varepsilon & \sim & \text{EV type 1, iid}
\end{eqnarray*}


\uncover<2->{
\begin{center}
\begin{tabular}{cc}
\textrm{$x_{it}$}: State vector $ $ & Meaning\tabularnewline
\hline 
\textcolor{red}{$a_{it}$} & \textcolor{red}{financial assets}\tabularnewline
$z_{it}$ & idiosyncratic income shock\tabularnewline
$s_{it}$ & household size: children present?\tabularnewline
\textcolor{red}{$p_{dt},q_{dt},\forall d$} & \textcolor{red}{regional prices}\tabularnewline
\textcolor{red}{$h_{it-1}$} & \textcolor{red}{housing status when entering period}\tabularnewline
$d$ & current region out of 9 in total\tabularnewline
\marktopleft{t1} $\tau$ & moving cost type \markbottomright{t1} \\
$t$ & age\tabularnewline
\hline 
\end{tabular}
\par\end{center}
}

\uncover<3->{\tikz[overlay,remember picture,inner sep=1pt]
\node[draw=red,ultra thick, rounded corners,fit=(marker-t1-a.north west) (marker-t1-b.south east)]{};
}

\end{frame}

\begin{frame}{Evidence on Moving Cost Types}
\begin{columns}
\begin{column}{0.5\textwidth}

\includegraphics[scale=0.35]{\Dropbox/pictures/transom-tweet}
\end{column}
\begin{column}{0.5\textwidth}
\begin{midi}
\item NYFed Consumer Survey
\item Subjects assign probabilities to hypothetical moves
\item Can compute WTP for types of moves.
\item \url{https://tyleransom.github.io/research/SCE_migration.pdf}
\end{midi}
\end{column}
\end{columns}
\end{frame}

\begin{frame}{Evidence on Moving Cost Types}
\begin{adjustbox}{center}
\includegraphics[scale=0.55]{\Dropbox/pictures/transom-cite}
\end{adjustbox}
\end{frame}

\begin{frame}{Housing and Savings Problem}

The housing and savings problem is defined as:
\begin{eqnarray*}
v_{t}\left(x_{it},d'\right) & = & \max_{c_{it}>0,h_{it}\in\left\{ 0,1\right\} }u(c_{it},h_{it},d_{it},d')\\
 & + & \beta\mathbb{E}_{z,s,\mathbf{q}_{t+1},\mathbf{p}_{t+1}}\left[\overline{v}_{t+1}\left(x_{it+1}\right)|z_{it},s_{it},\mathbf{q}_{t},\mathbf{p}_{t}\right]\\
s.t. &  & \text{budget constraints}
\end{eqnarray*}

\textrm{
\[
\overline{v}_{t+1}\left(x_{it+1}\right)=E_{\varepsilon}V_{t+1}\left(x_{it+1}\right)
\]
}
\end{frame}

\begin{frame}{Preferences: $u(c_{it},h_{it},d_{it},d')$}
\tikzstyle{na} = [baseline=-.5ex]

\begin{itemize}
\item The within-period payoff depends on 

\begin{itemize}
\item consumption $(\red{c})$, 
\item children present $(s)$, 
\item housing status $(h\in\{0,1\})$, 
\item whether move is made \textrm{$\left(\mathbf{1}\left[d\neq d'\right]\right)$,}
\item amenities ($A_{d'}$)
\item idiosyncratic location shock $\left(\varepsilon\right)$
\end{itemize}
\[
u\left(c_{it},h_{it},d,d'\right)=\eta \tikz[baseline]{
            \node[fill=red!20,anchor=base] (t1)
            {$ \frac{c_{it}^{1-\gamma}}{1-\gamma} $};
        } +\xi(s_{it})\times h_{it}-\mathbf{1}\left[d\neq d'\right]\Delta\left(x_{it}\right)+A_{d'}+\varepsilon_{id't}
\]

\item Moving Costs:
\[
\Delta(x)=\alpha_{0,\tau}+\alpha_{1}h_{it-1}+\alpha_{2}t_{it}+\alpha_{3}t_{it}^{2}+\alpha_{4}s_{it}
\]

\end{itemize}
\end{frame}


\begin{frame}{Preferences: $u(c_{it},h_{it},d_{it},d')$}
\begin{itemize}
\item The within-period payoff depends on 

\begin{itemize}
\item consumption $(c)$, 
\item children present $(\red{s})$, 
\item housing status $(h\in\{0,1\})$, 
\item whether move is made \textrm{$\left(\mathbf{1}\left[d\neq d'\right]\right)$,}
\item amenities ($A_{d'}$)
\item idiosyncratic location shock $\left(\varepsilon\right)$
\end{itemize}
\[
u\left(c_{it},h_{it},d,d'\right)=\eta \frac{c_{it}^{1-\gamma}}{1-\gamma} + 
\tikz[baseline]{
            \node[fill=red!20,anchor=base] (t1)
            {$ \xi(s_{it}) $};
        } \times h_{it}-\mathbf{1}\left[d\neq d'\right]\Delta\left(x_{it}\right)+A_{d'}+\varepsilon_{id't}
\]

\item Moving Costs:
\[
\Delta(x)=\alpha_{0,\tau}+\alpha_{1}h_{it-1}+\alpha_{2}t_{it}+\alpha_{3}t_{it}^{2}+\alpha_{4}
\tikz[baseline]{
            \node[fill=red!20,anchor=base] (t2)
            {$ s_{it}$};
            }
\]

\end{itemize}
\end{frame}

\begin{frame}{Preferences: $u(c_{it},h_{it},d_{it},d')$}
\begin{itemize}
\item The within-period payoff depends on 

\begin{itemize}
\item consumption $(c)$, 
\item children present $(s)$, 
\item housing status $(\red{h}\in\{0,1\})$, 
\item whether move is made \textrm{$\left(\mathbf{1}\left[d\neq d'\right]\right)$,}
\item amenities ($A_{d'}$)
\item idiosyncratic location shock $\left(\varepsilon\right)$
\end{itemize}
\[
u\left(c_{it},h_{it},d,d'\right)=\eta \frac{c_{it}^{1-\gamma}}{1-\gamma} + \xi(s_{it})\times
\tikz[baseline]{
            \node[fill=red!20,anchor=base] (t1)
            {$ h_{it}$};
        } -\mathbf{1}\left[d\neq d'\right]\Delta\left(x_{it}\right)+A_{d'}+\varepsilon_{id't}
\]

\item Moving Costs:
\[
\Delta(x)=\alpha_{0,\tau}+\alpha_{1}
\tikz[baseline]{
            \node[fill=red!20,anchor=base] (t2)
            {$ h_{it-1}$};
            }+\alpha_{2}t_{it}+\alpha_{3}t_{it}^{2}+\alpha_{4}s_{it}
\]

\end{itemize}
\end{frame}


\begin{frame}{Preferences: $u(c_{it},h_{it},d_{it},d')$}
\begin{itemize}
\item The within-period payoff depends on 

\begin{itemize}
\item consumption $(c)$, 
\item children present $(s)$, 
\item housing status $(h\in\{0,1\})$, 
\item whether move is made \textrm{$\left(\red{\mathbf{1}\left[d\neq d'\right]}\right)$,}
\item amenities ($A_{d'}$)
\item idiosyncratic location shock $\left(\varepsilon\right)$
\end{itemize}
\[
u\left(c_{it},h_{it},d,d'\right)=\eta \frac{c_{it}^{1-\gamma}}{1-\gamma} + \xi(s_{it})\times h_{it} - 
\tikz[baseline]{
            \node[fill=red!20,anchor=base] (t1)
            {$ \mathbf{1}\left[d\neq d'\right]\Delta\left(x_{it}\right) $};
        } +A_{d'}+\varepsilon_{id't}
\]

\item Moving Costs:
\[
\tikz[baseline]{
            \node[fill=red!20,anchor=base] (t2)
            {$ \Delta(x)$};
            } = \alpha_{0,\tau} 
+\alpha_{1} h_{it-1}+\alpha_{2}t_{it}+\alpha_{3}t_{it}^{2}+\alpha_{4}s_{it}
\]

\end{itemize}
\end{frame}


\begin{frame}{Preferences: $u(c_{it},h_{it},d_{it},d')$}
\begin{itemize}
\item The within-period payoff depends on 

\begin{itemize}
\item consumption $(c)$, 
\item children present $(s)$, 
\item housing status $(h\in\{0,1\})$, 
\item whether move is made \textrm{$\left(\red{\mathbf{1}\left[d\neq d'\right]}\right)$,}
\item amenities ($A_{d'}$)
\item idiosyncratic location shock $\left(\varepsilon\right)$
\end{itemize}
\[
u\left(c_{it},h_{it},d,d'\right)=\eta \frac{c_{it}^{1-\gamma}}{1-\gamma} + \xi(s_{it})\times h_{it} - 
\tikz[baseline]{
            \node[fill=red!20,anchor=base] (t1)
            {$ \mathbf{1}\left[d\neq d'\right]\Delta\left(x_{it}\right) $};
        } +A_{d'}+\varepsilon_{id't}
\]

\item Moving Costs:
\[
\tikz[baseline]{
            \node[fill=red!20,anchor=base] (t2)
            {$ \Delta(x)$};
            } = 
\tikz[baseline]{
            \node[fill=blue!20,anchor=base] (t3)
            {$ \alpha_{0,\tau}$};
            }  
+\alpha_{1} h_{it-1}+\alpha_{2}t_{it}+\alpha_{3}t_{it}^{2}+\alpha_{4}s_{it}
\]

\end{itemize}
\end{frame}


\begin{frame}{Preferences: $u(c_{it},h_{it},d_{it},d')$}
\begin{itemize}
\item The within-period payoff depends on 

\begin{itemize}
\item consumption $(c)$, 
\item children present $(s)$, 
\item housing status $(h\in\{0,1\})$, 
\item whether move is made \textrm{$\left(\mathbf{1}\left[d\neq d'\right]\right)$,}
\item amenities ($\red{A_{d'}}$)
\item idiosyncratic location shock $\left(\varepsilon\right)$
\end{itemize}
\[
u\left(c_{it},h_{it},d,d'\right)=\eta \frac{c_{it}^{1-\gamma}}{1-\gamma} + \xi(s_{it})\times h_{it} - 
\mathbf{1}\left[d\neq d'\right]\Delta\left(x_{it}\right) + 
\tikz[baseline]{
            \node[fill=red!20,anchor=base] (t1)
            {$ A_{d'} $};
        } + \varepsilon_{id't}
\]

\item Moving Costs:
\[
\Delta(x)=\alpha_{0,\tau}+\alpha_{1}h_{it-1}+\alpha_{2}t_{it}+\alpha_{3}t_{it}^{2}+\alpha_{4}s_{it}
\]

\end{itemize}
\end{frame}
\begin{frame}{Preferences: $u(c_{it},h_{it},d_{it},d')$}
\begin{itemize}
\item The within-period payoff depends on 

\begin{itemize}
\item consumption $(c)$, 
\item children present $(s)$, 
\item housing status $(h\in\{0,1\})$, 
\item whether move is made \textrm{$\left(\mathbf{1}\left[d\neq d'\right]\right)$,}
\item amenities ($A_{d'}$)
\item idiosyncratic location shock $\left(\red{\varepsilon}\right)$
\end{itemize}
\[
u\left(c_{it},h_{it},d,d'\right)=\eta \frac{c_{it}^{1-\gamma}}{1-\gamma} + \xi(s_{it})\times h_{it} - 
\mathbf{1}\left[d\neq d'\right]\Delta\left(x_{it}\right) + A_{d'}  +
\tikz[baseline]{
            \node[fill=red!20,anchor=base] (t1)
            {$\varepsilon_{id't} $};
        } 
\]

\item Moving Costs:
\[
\Delta(x)=\alpha_{0,\tau}+\alpha_{1}h_{it-1}+\alpha_{2}t_{it}+\alpha_{3}t_{it}^{2}+\alpha_{4}s_{it}
\]

\end{itemize}
\end{frame}


\begin{longv}
\begin{frame}{Housing Choice}{Rent or Own?}

\begin{midi}
\item $h_{t}=1$ indicates ownership in current period.
\item Owners may borrow with house as collateral, s.t. 
\[a_{t+1}\geq-(1-\chi)p_{dt}h_{t},\forall t\]
\item Terminal net wealth condition: $a_{T}+p_{dT}h_{T-1}\geq0$
\item No explicit repayment schedule.
\item Proportional cost of selling the house: $\phi$.
\end{midi}
\end{frame}
\end{longv}

\begin{frame}{Labor Income}

\begin{itemize}
\item $q_{dt}$ is an index of regional labor productivity.
\item Individual income is defined by:

\begin{eqnarray*}
\ln y_{idt} & = & \eta_{d}\ln q_{dt}+f(t)+z_{it}\\
z_{it} & = & \rho z_{it-1}+e_{it-1}\\
e & \sim & N(0,\sigma^{2})
\end{eqnarray*}

\item $f(t)$ is a deterministic age profile.
\item $z_{it}$ is an idiosyncratic productivity shock (AR1).
\end{itemize}
\end{frame}

\begin{frame}{Job Search: Labor income of movers}

\begin{midi}
\item People have to move to $d'$ before they discover the exact value
of new income: experience good.
\item Expectations over idiosyncratic $z$\textrm{ }differ for movers:\textrm{
\begin{eqnarray*}
 & E[z_{t+1}|z_{t},\text{stay}]\\
 & \neq\\
 & E[z_{t+1}|z_{t},\text{move}]
\end{eqnarray*}
}
\item Movers get a new draw $z_{t+1}$, correlated with their previous $z_{t}$,
according to the conditional distribution $G\left(z_{it+1}|z_{it}\right)$.
\item Rules out selection on offers.

\end{midi}
\end{frame}
%
\begin{frame}{Budget Constraints}

\begin{enumerate}
\item Choose to remain renters: pay rent $\kappa_{d}p_{dt}$
\begin{eqnarray*}
a_{it+1} & = & (1+r)\left(a_{it}+y_{idt}-c_{it}-\kappa_{d}p_{dt}\right)\\
a_{it+1} & \geq & 0
\end{eqnarray*}
\item \onslide+<2->Choose to buy:

\begin{eqnarray*}
a_{it+1} & = & (1+r(a_{it}))\left(a_{it}+y_{idt}-c_{it}-p_{dt}\right)\\
a_{it+1} & \geq & -(1-\chi)p_{dt}h_{it}\\
r(a_{it}) & = & \begin{cases}
r & \text{if }a_{it}\geq0\\
r^{m} & \text{if }a_{it}<0
\end{cases},r^{m}>r\\
p_{T}h_{iT-1}+a_{iT} & \geq & 0
\end{eqnarray*}

\item \onslide+<2->Skip rest in interest of time. \hyperlink{BudgetConstraints}{\beamergotobutton{No!}}\hypertarget{Budgets-origin}{}
\end{enumerate}
\end{frame}

\section{Estimation}
\begin{frame}[plain]{}

\vspace{1cm}

\begin{center}
\textbf{\textcolor{blue}{\LARGE{}Estimation}}{\LARGE\par}
\par\end{center}

\end{frame}

\begin{frame}{Data: Micro (SIPP) and Macro (BEA, FHFA)}

\begin{itemize}
\item Micro data:

\begin{itemize}
\item Use the 1996, 2001, 2004 and 2008 SIPP panels.
\begin{itemize}
\item 102,529 unique individuals, 2684 cross-Division moves.
\item construct moments as averages over 1997\textendash 2012 to obtain
reasonable sample size in conditioning cells.
\item Use a cohort estimator: accounts for fact that different cohorts faced
different aggregate shocks. 
\end{itemize}
\end{itemize}
\item Macro data:

\begin{itemize}
\item BEA personal income by state to get regional GDP/capita.
\item FHFA for regional house price indeces.
\end{itemize}
\end{itemize}
\end{frame}

\begin{frame}{Estimation Outline\hypertarget{CurseDim}{}}

Estimation proceeds in two steps:
\begin{enumerate}
\item Reduce dimensionality of stochastic processes: 

\begin{itemize}
\item An unrestricted model suffers from the curse of dimensionality: 20-dimensional
integration problem. \hyperlink{CurseDimDetail}{\beamergotobutton{details}}
\item I will employ a low-dimensional factor model to approximate this space.
\end{itemize}
\item Estimate the decision model:

\begin{itemize}
\item I will fix some parameters, and
\item Estimate the remaining behavioural parameters with SMM.
\end{itemize}
\end{enumerate}
\end{frame}

\begin{longv}
\begin{frame}{Identification}

\begin{midi}
\item I use observed variation in regional prices, productivity levels, migration rates, ownership patterns and migration flows to identify the model.
\item Identifying assumptions: 
\begin{enumerate}
\item Individual housing demand is not subject to aggregate shocks.
\item Preferences are identical across regions (net of the regional fixed effect).
\end{enumerate}
\pause
\item Implications:
\begin{enumerate}
\item For estimation purposes, I can take $p_{dt},q_{dt}$ as exogenous to individual decisions: household $i$ responds to changes in $p_{dt},q_{dt}$ across regions and time.
\item I implicitly assume that $p_{dt},q_{dt}$ are shifted by aggregate shocks only.
\end{enumerate}
\end{midi}

\end{frame}


\end{longv}

\begin{frame}{Step 1: Factor Model of Aggregate and Regional Prices }

\begin{midi}
\item I assume that agents use a 2-dimensional factor model to forecast
prices.
\item An aggregate factor $\mathbf{F}_{t}\equiv\left[Q_{t},P_{t}\right]$
follows a stationary VAR(1).
\item $\mathbf{a}_{d}$ is a known matrix of factor loadings.
\item Regional state variables $q_{dt}$ and $p_{dt}$ are known functions
of $\mathbf{F}_{t}$ and $\mathbf{a}_{d}$:
\[
\left[q_{dt},p_{dt}\right]=g\left(\mathbf{F}_{t},\mathbf{a}_{d}\right)
\]
\item This is similar to \cite{krusell_smith}.
\end{midi}

\end{frame}
\begin{frame}[plain]{}

\begin{center}
\includegraphics[height=0.91\paperheight]{\Dropbox/data/sipp/VAR_reg_p}
\par\end{center}

\end{frame}

\begin{frame}[plain]{}

\begin{center}
\includegraphics[height=0.91\paperheight]{\Dropbox/data/sipp/VAR_reg_y}
\par\end{center}

\end{frame}

% \begin{frame}{Step 2: Preset Parameters}

% \hspace*{-2pt}\makebox[\linewidth][c]{
% \renewcommand{\arraystretch}{1.2}  
% \input{\Dropbox/model/fit/preset.tex}
% }
% \end{frame}

\begin{longv}
\begin{frame}{Step 2: SMM}
\begin{midi}
\item We work with parameter vector \[\theta=\left\{ \alpha_{0},\alpha_{1},\alpha_{2},\alpha_{3},\alpha_{4},\omega,\pi_{\tau},\eta,\xi_{1},\xi_{2}\right\} \]
\item I use an SMM objective function
    \[L(\theta)=\left[m-\hat{m}(\theta)\right]^{T}W\left[m-\hat{m}(\theta)\right]\]
\item Optimized using a cyclic coordinate search algorithm.
\item $\mapsto$ Check it out! \texttt{optSlices()} in \url{https://github.com/floswald/MomentOpt.jl}
\end{midi}
\end{frame}
\end{longv}


\begin{frame}{Step 2: Key Moment Relationships}

\hspace*{-2pt}\makebox[\linewidth][c]{%
\renewcommand{\arraystretch}{1.2}  

\begin{tabular}{ccc}
Parameter & Description & Moment\tabularnewline
\hline 
$\alpha_{0}$ & Moving Cost & Average mobility\tabularnewline
$\alpha_{1}$ & Moving Cost Owners & Owner mobility\tabularnewline
$\alpha_{2}$ & Moving Cost age & age profile of mobility\tabularnewline
$\alpha_{3}$ & Moving Cost age & age profile of mobility\tabularnewline
$\alpha_{4}$ & Moving Cost kids & mobility cond. on kids\tabularnewline
$A_{d'}$ & Amenity in $d'$ & Proportion of flows to $d'$\tabularnewline
$\omega$ & Continuation value & fraction of owners who sell age 50\tabularnewline
$\pi_{\tau}$ & proportion of high cost & proportion who never move\tabularnewline
$\xi_{1},\xi_{2}$ & utility from owning & ownership profile\tabularnewline
\hline 
\end{tabular}

}
\end{frame}

\begin{longv}


\begin{frame}{Step 2: Full List of Moments}

\begin{itemize}
\item Moments relating to mobility:
\begin{itemize}

\item $E[\text{move}]$, $E\left[\text{move to }d'\right]$, $E[\text{move}|T]$,
$E[\text{move}|\text{kids}]$, $E[\text{move}|h]$, $Cov(\text{move},\text{kids})$,
$Cov(\text{move},h)$, $E[\text{moved never}]$, $E[\text{moved once}]$,
$E[\text{moved twice+}]$, 
\item Auxiliary model
\[
\text{move}_{it}=\beta_{0,m}+\beta_{1,m}t_{it}+\beta_{2,m}t_{it}^{2}+u_{it}
\]
\end{itemize}
\item Moments relating to homeownership:

\begin{itemize}
\item $E[h]$, $E[h|d]$, $E[h|\text{kids}]$, $E[h_{t-1}=1,h_{t}=0|T]$,
$Cov(h,\text{kids})$
\item Auxiliary model
\[
h_{it-1}=\beta_{0,h}+\beta_{1,h}t_{it}+\beta_{2,h}t_{it}^{2}+u_{it}
\]
\end{itemize}
\hypertarget{moments}{}
\begin{flushright}
\hyperlink{param-estimates}{\beamergotobutton{Details.}}\par
\end{flushright}

\end{itemize}
\end{frame}
\end{longv}


\begin{frame}{Model Fit}

\hspace*{-2pt}\makebox[\linewidth][c]{%

\includegraphics[scale=0.5]{\Dropbox/model/fit/fit_mobility}\includegraphics[scale=0.5]{\Dropbox/model/fit/fit_ownership}

}

\end{frame}

\begin{frame}{Model Fit }

\hspace*{-2pt}\makebox[\linewidth][c]{%

\includegraphics[scale=0.5]{\Dropbox/model/fit/fit_auxmods2}\includegraphics[scale=0.5]{\Dropbox/model/fit/fit_wealth}

}
\end{frame}


\section{Value of migration }



\begin{frame}[plain]{}

\vspace{1cm}

\begin{center}
\textbf{\textcolor{blue}{\LARGE{}Results}}{\LARGE\par}
\par\end{center}

\end{frame}


\begin{frame}{Elasticity of Migration wrt Income Shock}

\begin{midi}
\item Region $j$ gets unexpected and permanent $+1\%$ shock to $q_j$.
\item What's the percentage change in stocks and flows of population?
\item By region in paper, here I report average across regions.
\end{midi}
\pause
\vspace{0.5cm}

\begin{adjustbox}{center}
\input{\Dropbox/model/fit/elasticities_q1.tex}
\end{adjustbox}
\end{frame}

\begin{frame}{Elasticity of Migration wrt Income Shock}

\begin{midi}
\item Region $j$ gets unexpected and permanent $+1\%$ shock to $q_j$.
\item What's the percentage change in stocks and flows of population?
\item By region in paper, here I report average across regions.
\end{midi}

\vspace{0.5cm}
\begin{adjustbox}{center}
\input{\Dropbox/model/fit/elasticities_q2.tex}
\end{adjustbox}
\end{frame}

\begin{frame}{Elasticity of Migration wrt Income Shock}

\begin{midi}
\item Region $j$ gets unexpected and permanent $+1\%$ shock to $q_j$.
\item What's the percentage change in stocks and flows of population?
\item By region in paper, here I report average across regions.
\end{midi}

\vspace{0.5cm}
\begin{adjustbox}{center}
\input{\Dropbox/model/fit/elasticities_q0.tex}
\end{adjustbox}
\end{frame}

\begin{frame}{Who Do Owners Move Less?}

\begin{midi}
\item Owners have additional moving costs $\alpha_3$.
\item There are fixed selling costs $\phi$.
\pause
\item What's the importance of each of those components?
\item Investigate quantitative implications of removing each of them.
\pause
\item (This kind of approximates $\frac{\partial m}{\partial \alpha_3},\frac{\partial m}{\partial \phi}$, $m$ the moment function.)
\end{midi}

\end{frame}

\begin{frame}{Why Do Owners Move Less?}

\begin{table}
\centering{}\input{\Dropbox/model/experiments/decompose_MC/decompose_MC0.tex}
\end{table}
\end{frame}

\begin{frame}{Why Do Owners Move Less?}

\begin{table}
\centering{}\input{\Dropbox/model/experiments/decompose_MC/decompose_MC1.tex}
\end{table}
\end{frame}

\begin{frame}{Why Do Owners Move Less?}

\begin{table}
\centering{}\input{\Dropbox/model/experiments/decompose_MC/decompose_MC2.tex}
\end{table}
\end{frame}

\begin{frame}{Why Do Owners Move Less?}

\begin{table}
\centering{}\input{\Dropbox/model/experiments/decompose_MC/decompose_MC.tex}
\end{table}
\end{frame}

\begin{frame}{Owner Regret?}
\begin{midi}
\item Buying a house is partly based on expectations. Model agents expect \red{aggregate} shocks to be transmitted via $\mathbf{a}_j$ to their region $j$.
\item \onslide+<2-> What if things turn out \bblue{not} as expected?
\item \onslide+<3-> Suppose that $\mathbf{a}_j$ unexpectedly changes to $\mathbf{a'}_j$.
\item \onslide+<4-> How much would owner in this \bluebox{broken promise} scenario pay to \bred{reset} to a zero networth renter? 
\end{midi}

\end{frame}

\begin{frame}{Owner Regret?}


\begin{table}
\begin{adjustbox}{center}
\input{\Dropbox/model/experiments/ownersWTP2.tex}
\end{adjustbox}
\caption{In 1000 of Dollars.}
\end{table}
\end{frame}


\begin{frame}{The Option Value of Migration}{How much consumption would you forgo to be able to migrate?}

\begin{midi}
\item \cite{yagan} talks about \blue{migratory insurance} in 2006 US recession: ex-post, how many workers moved from worst-affected areas? 7\%.
\item \onslide+<2-> Here: \cyanbox{Ex-ante}, given you expect shocks, how much would you pay for an insurance policy that allows you to leave whenever you want.
\item \onslide+<3-> Compare two model versions: baseline and complete migration shutdown scenario.
\end{midi}

\end{frame}

\begin{frame}{The Option Value of Migration}

\begin{adjustbox}{center}
\input{\Dropbox/model/experiments/noMove_region_z_ctax1.tex}
\end{adjustbox}

\end{frame}

\begin{frame}{The Option Value of Migration}

\begin{adjustbox}{center}
\input{\Dropbox/model/experiments/noMove_region_z_ctax2.tex}
\end{adjustbox}

\end{frame}
\begin{frame}{The Option Value of Migration}

\begin{adjustbox}{center}
\input{\Dropbox/model/experiments/noMove_region_z_ctax3.tex}
\end{adjustbox}

\end{frame}
\begin{frame}{The Option Value of Migration}

\begin{adjustbox}{center}
\input{\Dropbox/model/experiments/noMove_region_z_ctax4.tex}
\end{adjustbox}

\end{frame}


\begin{frame}{Conclusion}

\begin{widee}
\item People greatly value regional migration: 19\% of period consumption.
\item \bblue{Owner}/\bred{Renter} margin is first order: Moving Elasticities are orders of magnitude apart
\item This has important implications for housing and labor market policies alike.
\end{widee}

\end{frame}

\begin{frame}[plain]{}

\vspace{1cm}

\begin{center}
\textbf{\textcolor{blue}{\LARGE{}Thanks for listening!}}{\LARGE\par}
\par\end{center}

\end{frame}


\begin{frame}{Parameter estimates}

\begin{center}
\hypertarget{param-estimates}{}
\input{\Dropbox/model/fit/params.tex}\par
\par\end{center}

\end{frame}

\begin{frame}{The Curse of Dimensionality: Regional Prices.\hypertarget{CurseDimDetail}{}}

\begin{itemize}
\item If $q_{dt}$ and $p_{dt}$ are unrestricted Markovian processes, computing
the expectation of $\overline{v}(x)$ is a 20-dimensional integration
problem (integrate over $z,s,\mathbf{q},\mathbf{p}$).
\begin{itemize}
\item Assuming 3 points for each grid $q_{d}$ and $p_{d}$, this implies
$3^{18}$ points for $\mathbf{q},\mathbf{p}$ alone.
\item If there are $N$ other states, we have $N\times3^{18}$ points in
the state space
\item In current version of the model, $N=156.672$, thus we would have

\[
60.697.942.852.608
\]
points to compute one solution. This is currently infeasible.
\end{itemize}
\hyperlink{CurseDim}{\beamergotobutton{back}}
\end{itemize}
\end{frame}

%
\begin{frame}{Budget Constraint: Renters who decide to stay\hypertarget{BudgetConstraints}{}
\hyperlink{Budgets-origin}{\beamergotobutton{back}}}
\begin{enumerate}
\item Choose to remain renters: pay rent $\kappa_{d}p_{dt}$
\begin{eqnarray*}
a_{it+1} & = & (1+r)\left(a_{it}+y_{idt}-c_{it}-\kappa_{d}p_{dt}\right)\\
a_{it+1} & \geq & 0
\end{eqnarray*}
\item \onslide+<2->Choose to buy:
\end{enumerate}
\begin{eqnarray*}
a_{it+1} & = & (1+r(a_{it}))\left(a_{it}+y_{idt}-c_{it}-p_{dt}\right)\\
a_{it+1} & \geq & -(1-\chi)p_{dt}h_{it}\\
r(a_{it}) & = & \begin{cases}
r & \text{if }a_{it}\geq0\\
r^{m} & \text{if }a_{it}<0
\end{cases},r^{m}>r\\
p_{T}h_{iT-1}+a_{iT} & \geq & 0
\end{eqnarray*}

\end{frame}
%
\begin{frame}{Budget Constraint: Owners who decide to stay}

\begin{itemize}
\item No fixed mortgage payment: implied schedule.
\item Continue as owners:

\end{itemize}
\begin{eqnarray*}
a_{it+1} & = & (1+r(a_{it}))\left(a_{it}+y_{idt}-c_{it}\right)\\
a_{it+1} & \geq & -(1-\chi)p_{dt}\\
p_{T}h_{iT-1}+a_{iT} & \geq & 0
\end{eqnarray*}

\begin{itemize}
\item \onslide+<2->Owners can choose to sell, incurring proportional fixed
cost $\phi$:

\end{itemize}
\begin{eqnarray*}
a_{it+1} & = & (1+r)\left(a_{it}+y_{idt}-c_{it}+(1-\phi)p_{dt}-\kappa_{d}p_{dt}\right)\\
a_{it+1} & \geq & 0
\end{eqnarray*}

\end{frame}

\begin{frame}{Budget Constraint: Movers}

\begin{itemize}
\item When moving from $d$ to $d'$, households can either rent or buy
in the new location $d'$.
\begin{itemize}
\item Renters in $d$:

\begin{itemize}
\item rent in $d'$: $a_{it+1}=(1+r)\left(a_{it}+y_{idt}-c_{it}-\kappa_{d'}p_{d't}\right)$
\item \onslide+<2->buy in $d'$: $a_{it+1}=(1+r(a_{it}))\left(a_{it}+y_{idt}-c_{it}-p_{d't}\right)$
\end{itemize}
\item \onslide+<3->Owners in $d$:

\begin{itemize}
\item rent in $d'$: 
\[
a_{it+1}=(1+r)\left(a_{it}+y_{idt}-c_{it}+(1-\phi)p_{dt}-\kappa_{d'}p_{d't}\right)
\]
\item \onslide+<4->buy in $d'$: 
\[
a_{it+1}=(1+r)\left(a_{it}+y_{idt}-c_{it}+(1-\phi)p_{dt}-p_{d't}\right)
\]
\end{itemize}
\end{itemize}
\end{itemize}
\end{frame}



\begin{frame}[allowframebreaks]{References}
        % \frametitle{References}
		\bibliographystyle{unsrtnat}
		\bibliography{../lyx/references.bib}
\end{frame}

\end{document}
