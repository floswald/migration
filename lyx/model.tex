%% LyX 2.3.2 created this file.  For more info, see http://www.lyx.org/.
%% Do not edit unless you really know what you are doing.
\documentclass[11pt,english]{article}
\usepackage[T1]{fontenc}
\usepackage[latin9]{inputenc}
\usepackage[a4paper]{geometry}
\geometry{verbose,lmargin=2cm,rmargin=2cm}
\setlength{\parskip}{\smallskipamount}
\setlength{\parindent}{0pt}
\usepackage{color}
\usepackage{babel}
\usepackage{float}
\usepackage{url}
\usepackage{amsmath}
\usepackage{amssymb}
\usepackage{graphicx}
\usepackage{setspace}
\onehalfspacing
\usepackage[unicode=true,pdfusetitle,
 bookmarks=true,bookmarksnumbered=false,bookmarksopen=false,
 breaklinks=true,pdfborder={0 0 0},pdfborderstyle={},backref=false,colorlinks=true]
 {hyperref}

\makeatletter
%%%%%%%%%%%%%%%%%%%%%%%%%%%%%% User specified LaTeX commands.
\usepackage{natbib}
\usepackage{url}
\usepackage{booktabs}
\usepackage{dcolumn}
\usepackage{longtable}
\usepackage{xcolor}
\definecolor{dark-red}{rgb}{0.8,0.15,0.15}
\definecolor{dark-blue}{rgb}{0.15,0.15,0.8}
\definecolor{medium-blue}{rgb}{0,0,0.5}
\hypersetup{
    colorlinks, linkcolor={dark-red},
    citecolor={dark-blue}, urlcolor={medium-blue}
}
% \newcommand{\Dropbox}{${HOME}/Dropbox/research/mobility/output}
\newcommand{\Dropbox}{repl-box/output}
\usepackage{tikz}
\usepackage{adjustbox}
\usepackage{pgfplots}
\usepackage{makecell}
\usepackage{xr}
\externaldocument[R-]{model-online}

\makeatother

\begin{document}
\title{The Effect of Homeownership on the Option Value of Regional Migration\thanks{I would like to thank Costas Meghir, Lars Nesheim and Mariacristina
De Nardi for their continued support and advice throughout this project.
I also thank Uta Sch�nberg, Jeremy Lise, Marco Bassetto, Suphanit
Piyapromdee, Richard Blundell, Jose-Victor Rios Rull, Morten Ravn, 
Fabien Postel-Vinay, Emeric Henry as well as three anonymous referees for helpful comments and suggestions. I would
like to acknowledge financial support from cemmap under ESRC grant
ES/I034021/1 and ESRC transformative grant ES/M000486/1.}}
\author{Florian Oswald\thanks{email: \protect\url{florian.oswald@gmail.com}. }\\
SciencesPo Paris}
\date{\today}
\maketitle
\begin{abstract}
This paper estimates a lifecycle model of consumption, housing choice
and migration in the presence of aggregate and regional shocks, using
the Survey of Income and Program Participation (SIPP). The model delivers
structural estimates of moving costs by ownership status, age and
family size that complement the previous literature. Using the model
I first show that migration elasticities vary substantially between
renters and owners, and I estimate the consumption value of having
the option to migrate across regions when there are regional shocks.
This value is 19\% of lifetime consumption on average, and it varies
substantially with household type.
\end{abstract}
\textbf{JEL-Codes: }J6, R23, R21

\vfill


\section{Introduction}

Regional migration rates in the USA are relatively low despite the
presence of large regional shocks. However, it would be a mistake
to conclude from this observation that the option to migrate across
regions has a small value to consumers. The goal of this paper is
to provide a measure of having the option to migrate in the face of
regional income and house price uncertainty, and I show that the value
is large. The paper provides a structural interpretation of the insurance
value of migration against regional shocks, as proposed first in \cite{yagan}.
It shows that considering homeowners and renters separately is of
first order importance for this issue, since both have vastly different
elasticities of migration with respect to regional shocks. This insight
is relevant for labor market and housing policy alike.

Migration probabilities are heterogeneous in the population. Which
type a of household is likely to move in a regional downturn? In this
paper, which is among the first to consider homeownership and migration
in an empirical lifecycle model, I provide structural estimates of
crucial objects related to this question, for example, moving costs
by ownership status, age and other observables. Modelling homeownership
realistically requires modelling asset accumulation and mortgages,
and it requires a proper treatment of expectations about house prices,
both of which are computationally demanding to integrate in a dynamic
model of location choice.

Homeownership and geographical mobility of households are tightly
connected: Renters are more mobile than owners. What complicates the
analysis, however, is that renters may choose to be renters precisely
\emph{because} they\emph{ }are more mobile, in the sense that they
might assess their own likelihood of moving to be relatively high.
What is more, often the econometrician cannot observe the relevant
state variables which would be informative about those considerations,
hence, there is unobserved heterogeneity at play. The model introduced
below allows to resolve the joint determination of housing tenure
status, consumption, savings and mobility decisions, such that it
can be used to structurally estimate deep parameters and to investigate
counterfactual policies.

The main counterfactual will be used to shed light on the option value
of regional migration under regional price and income risk. How much
would households want to pay for a hypothetical migration insurance
policy, in other words, what is the value of the option to move? In
order to address this, the experiment shuts down migration in the
economy, and it reports the compensating consumption stream which
would make individuals indifferent between this regime, and the status
quo, that is, a world with migration. The results of this exercise
differ greatly by type of household considered and their respective
current locations.

In 2013, 63\% of occupied housing units in the US were owned, while
37\% were rented.\footnote{see American Community Survey 2013, table DP04.}
At the same time, roughly 1.3\% of the population migrate across US
Census Division boundaries per year. Conditional on ownership, this
implies that 1.9\% of renters and 0.67\% owners move. A natural question
is then to ask why do we observe owners moving less? All else equal,
owners face higher moving costs, both in terms of financial as well
as time and effort costs. Financial costs occur because of transaction
costs in the housing market upon sale of the house (e.g. agency fees
or transaction taxes), while costs of effort arise from owners having
to spend time finding a suitable buyer, meet with agents and lawyers
etc. A comparable renter is subject to those costs only to a lesser
degree. Buying a house means to make a highly local financial investment,
which is subject to shocks as discussed above, is relatively illiquid,
and in addition may have a location specific flow of utility. Consumers
may have preferences for locations. Finally, as already mentioned,
there is selection into homeownership based on unobservable moving
costs: Individuals with a particular distaste for moving will be more
likely to select into homeownership, because they anticipate that
they are unlikely to ever move in the future. All of these factors
interact to shape the joint decision of housing tenure, location choice,
and mortgage borrowing. What is more, they all interact to influence
the decision to move in response to a shock.

In the model I develop, there are several mechanisms which affect
the home ownership choice of individuals. A downpayment requirement
for implies means that only individuals with sufficient cash on hand
are able to buy a house at the current price. The model assumes a
preference for owner-occupied accomodation, a local amenity and a
partially unobserved cost of moving, which influence the buying decision
in addition to age, the probability of moving, and beliefs about future
shocks. 

In terms of the decision to migrate to another region, the model predicts
that the likelihood of migration is increasing in the difference of
discounted expected lifetime utilities between any two regions. Those
relative utilities, in turn, depend among other things on the average
regional income level and the level of regional house prices, both
of which vary over time. Allowing regional characteristics to vary
is a significant contribution to the literature on dynamic migration
models such as for example \cite{kennan2003effect}, since it provides
a fundamental reason for agents to move in response to a change in
their economic environment, rather than as a result of idiosyncratic
preference shocks alone. Including time-varying location characteristics,
however, increases computational demands substantially. To keep those
demands tractable, the model employs a factor structure which allows
aggregate shocks to affect regions differently.

I estimate the model using a simulated method of moments estimator.
I find that the model fits the data very well along the main dimensions
of interest, which are mobility and ownership patterns over the lifecycle,
ownership rates by region, migration flows across regions, as well
as wealth accumulation over the lifecycle and by region. After fitting
the model to the data, I first use the model to compute migration
elasticities to regional shocks by tenure status and current location.
Then I investigate why owners move less than renters in greater detail.
The main result of the paper shows that migration is a low probability
event in both data and model, but associated with a large option value
for consumers. Shutting down regional migration in an environment
with realistic income and price shocks would require a 19\% increase
in per period consumption to make the average consumer indifferent
to the status quo. This number varies greatly by household type (age,
housing tenure, persistent income level) as well as location.

\paragraph{Literature. }

My paper builds on \cite{kennan2003effect}, who are the first to
develop a model of migration with multiple location choices over the
lifecycle. Their main finding is that expected income is an important
determinant of migration decisions, and their framework requires large
moving costs to match observed migration decisions. The model features
location-specific match effects in wages and amenity which are uncertain
ex-ante, so the consumer has to move to a location in order to discover
their values. The distributions of those match effects in each location
are stationary. After having learned the value of the current location,
the only reason for a move is a favourable realization of an i.i.d.
preference shock which might occur in some future period. There is
no change in economic fundamentals which might encourage a move, like
a shock to wages, for example. Relaxing this feature as well as adding
housing and savings decisions are my main contributions to their paper.
I am able to let regions experience differential income and price
shocks over time, thereby providing an additional reason to move over
and above idiosyncratic shocks. 

\cite{gemici2007family} focuses on migration decisions of couples
with two working spouses and finds that, for this subgroup, family
ties can significantly hinder migration decisions and wage growth.
\cite{winkler2010effect} is similar to \cite{gemici2007family} but
with housing choices. The main differences to \cite{winkler2010effect}
are the way I model regional price and income dynamics and the assumption
about how job search takes place. Regarding regional dynamics, I am
able to allow for shocks which are correlated across regions and with
an aggregate component that is persistent, while they are assumed
to be independent in \cite{winkler2010effect}. The i.i.d. assumption
for regional shocks is clearly rejected in the data, as will become
clear in the next section. Also, \cite{winkler2010effect} assumes
that job offers arrive in the current location from a random alternative
location. My assumption implies firstly that individuals consider
all potential locations in each period, and decide to move based on
their expectations about how they will fare in each. Secondly it allows
for reasons other than job offers to trigger a move, which is also
a feature of the data, as I will show below. \cite{ransom2018labor}
is another related paper using the \cite{kennan2003effect} setup
which allows for shocks to wages and local unemployment rates at the
CBSA level, but without considering housing. Finally, \cite{Bishop:2008fk}
computes a dynamic migration model using the conditional choice probability
setup as proposed by \cite{arcidiacono2008ccp} in order to recover
willingness to pay for environmental amenities.

By considering regional shocks, the present paper is related to the
seminal contribution of \cite{blanchard1992regional}. In light of
state-specific shocks to labor demand, the authors find that after
an adverse shock, the relocation of workers is one of the main mechanisms
to restore unemployment and participation rates back to trend in an
affected region. \cite{lkhagvasuren2012big} is a more recent paper
on the topic, proposing a frictional version of the \cite{lucas1974equilibrium}
island model. Relative to those papers, here we show how the underlying
decision maker reacts to regional shocks \textendash{} in particular,
how owners and renters react differently and what this implies for
their valuation of the migration option. Related to this, \cite{notowidigdo2011incidence}
analyses the incidence of local labor demand shocks on low-skilled
workers in a static spatial equilibrium model and finds that they
are more likely to stay in a declining city than high-skilled workers
to take advantage of cheaper housing.\footnote{See \cite{Moretti:2011fk} for a comprehensive overview of this literature
going back to \cite{Roback:1982uq} and \cite{rosen1979wage}, and
\cite{diamond2016determinants} and \cite{piypromdee} for recent
applications.} The same mechanism operates in my model. Furthermore, the dynamic
nature of my model allows me to evaluate the response of migration
to shocks over time. The present paper can be seen as a complement
to the exercise proposed in \cite{yagan}, or \cite{yagan2017employment},
where the question is how much insurance against local labor market
shocks is offered by migration. The author finds migration insures
against 7\% of an average local labor demand shock. I implement a
fully structural analysis of the same question, with the added benefit
that I can measure a value of the migration option in terms of consumption.
In this sense, the present paper offers a more direct answer to the
question of \emph{how much consumption would I forgo today in order
to be insured in an adverse future state}, which describes an insurance
contract fairly well. \cite{bartik2018moving} is a recent
paper which extends \cite{yagan} to consider the influence of the
China trade shock as well as the Fracking boom, abstracting from a detailed model of housing.

Another related literature considers the effects of the 2007 housing
bust on labor market mobility. In terms of empirical contributions,
\cite{Ferreira201034}, \cite{schulhofer2011negative} and \cite{demyanyk2013moving}
look at whether negative equity in the home reduces the mobility of
owners and report mixed findings. The first paper finds an effect,
whereas the next two do not, with the difference arising from different
datasets and definitions of long-distance moves. More theoretical
papers like \cite{head2012housing}, \cite{nenov2012regional}, \cite{csahin2014mismatch}
and \cite{karahan2011housing} use search models of labor and housing
markets to look at geographical mismatch in order to understand how
a fall in house prices affects unemployment and migration rates. The
last paper, in particular, formalizes the negative equity lock-in
notion in a model with two locations and finds only a moderate effect
of lock\textendash in on the increase in unemployment. The present
paper differs from this group of contributions by assuming multiple
locations and by adopting a life-cycle framework.\footnote{In general, the relationship between homeownership and labor market
mobility or unemployment has been discussed in many other places,
and an incomplete list might include \cite{oswald1996conjecture,blanchflower2013does},
\cite{coulson2002tenure}, \cite{guler2011homeownership}, \cite{battu2008housing}
or \cite{halket2014saving}.}

In the remainder of this papers I will first present a set of facts
from aggregate and micro data about regional migration in the US in
section \ref{sec:Facts} before introducing a structural model which
can speak to those fact in section \ref{sec:Model}. I will then discuss
solution and estimation of the model in sections \ref{sec:Solving-and-Simulating}
and \ref{sec:Estimation} in order to finally present the results
regarding the option value of migration in section \ref{sec:Results}.

\section{Facts \label{sec:Facts}}

According to \cite{molloy2011internal}, who use three publicly available
datasets (American Community Survey (ACS), the Annual Social and Economic
Supplement to the CPS (March CPS), and Internal Revenue Service (IRS)
data), each year roughly 5\% of the population moves between counties
each year, which amounts to roughly one-third of the annual flows
into and out of employment according to the measure in \cite{fallick2004employer}.
The cross State figure is 2\%, and the cross Census Division rate
is estimated at 1.5\% of the population, per year (see table \ref{tab:Migration-rates-Molloy}).

\begin{table}
\begin{centering}
\input{\Dropbox/data/Molloy/molloy.tex}
\par\end{centering}
\caption{Percent of US population migrating across different geographic boundaries
over different time spells. Taken from \cite{molloy2011internal},
computed from ACS, March CPS and IRS data.\label{tab:Migration-rates-Molloy}}
\end{table}

It is somewhat unfortunate that none of the datasets employed by \cite{molloy2011internal}
are well suited for the purpose of analysing migration and ownership.
None of them tracks movers, so it is impossible to know the circumstances
of an individual at the moment they decided to move, which is ultimately
of interest in this paper.\footnote{It is possible to construct a panel dataset from the CPS, but only
with postal address as unit identifier. If an individual moves out,
this can be inferred from the data, however, the destination of the
move cannot \textendash{} in particular it is unknown whether they
relocated withing the city, or somewhere else.} I therefore use the Survey of Income and Program Participation (SIPP)
in this paper, a longitudinal and nationally representative dataset.\footnote{The PSID is a natural competitor to the SIPP for this kind of study,
with the PSID's main advantage being the fact it's a long panel. I
found that cell sizes got extremely small, however, after conditioning
on the most important covariates in the PSID. Even unconditionally
there are only 1560 unique cross-Division moves in the PSID 1994\textendash 2011,
and four cells in the region-by-region transition matrix have no observations
for this entire period. I have 2512 unique cross-Division moves in
SIPP 1996\textendash 2012 and the corresponding transition matrix
is dense.} 

Before presenting statistics from SIPP data, I will explain the geographic
concept I will be using in this paper, which is a US Census Division.
Census Divisions are nine relatively large regions which separate
the United States into groups of states ``for the presentation of
census data\emph{''}\footnote{See the Census bureau's website at \url{https://www.census.gov/geo/reference/gtc/gtc_census_divreg.html}. }\emph{.
}To a first approximation, those regions represent areas with a common
housing and labor market. In the model, a move within any region is
not considered as migration and therefore does not contribute to the
overall migration rate. This implies that there is a proportion of
moves across markets that do happen in the data, but which are not
picked up by my geographic definition of a market. 

The aggregation of states into this particular grouping is but one
of many possibilities, and I adopt this particular partition based
on computational constraints. In many respects the ideal concept of
a region is what economists would refer to as a local labor market,
and metropolitan statistical areas (MSA) or commuting zones (CZ) come
close to this. Unfortunately, for the purpose of the model in this
paper, the so\textendash defined number of regions would be far too
large to be computationally feasible. Hence the choice of census divisions.\footnote{The model presented below contains 25.4 million different points in
the state space at which to solve a savings problem. Increasing the
number of regions to 51 (to represent US states) increases this to
815 million points in the state space. Given that estimation requires
evaluation of the model solution many times over, the former state
space can be handled with code that is highly optimized for speed,
while the latter cannot.} I will demonstrate below what the choice of Divisions implies for
the captured state\textendash level variation. In the online appendix
figure \ref{R-fig:Census-Division-Map} presents a map, and table
\ref{R-tab:SIPP-divisions} lists Division abbreviations and the member
States.\footnote{The online appendix is available at \url{http://qeconomics.org} and \url{https://floswald.github.io/pdf/homeownership-appendix.pdf}}

\subsection{The Main Reasons to Move are: Work, Housing and Family}

The March Supplement to the Current Population Survey (CPS) contains
several questions relevant for the study of migration. Here I analyse
answers to the 2013 edition of the CPS to the question ``What was
the main reason for moving'' where respondents are offered 19 options
to choose from. The results are displayed in table \ref{tab:CPS-reason-move}.
It is striking to note that even though we are conditioning on moves
across Division boundaries (and thus think of long-range moves), the
percentage of people citing category ``housing'' as their main motivation
is roughly 24\% of the total population of movers. The table also
disagreggates the response to the question by the distance between
origin and destination State, and we can see that the proportion of
respondents does vary with distance moved, but not to an extent that
would suggest that housing becomes irrelevant as a motivation with
increasing distance. Summing up in the bottom row of the table, we
see that 55\% say work was the main reason, 24\% refer to housing
and the remaining 21\% is split between family and other reasons.
The model to be presented below addresses each of these categories:
Individuals can move out of work-related concerns (regional and individual
level income fluctuations), because of housing considerations (regional
house price fluctuation), for family reasons (stochastic age-dependent
arrival of children) as well reasons classified as ``other'', which
are accounted for by an idiosyncratic preference shock.

\subsection{Homeownership and College Education are Important Predictors for
Migration}

Putting somewhat more structure onto this, I next present estimates
from a statistical analysis of the determinants of cross division
moves from household\textendash level SIPP data. I combine four panels
of SIPP data (1996, 2001, 2004 and 2008) into a database with 102,529
household heads that I can follow over time and space. This will be
the central estimation sample in main analsis below. Table \ref{tab:D2D-probit}
shows the results.\footnote{It's worth emphasizing that at this point I am abstracting away from
the severe endogeneity issues which the structural model below will
account for.} I regress a binary indicator for whether or not a cross division
move took place in a given year on a set of explanatory variables,
which relate to the household in question in a probit regression.
The table shows marginal effects computed at the sample mean of each
variable, as well as the ratio of marginal effects to the baseline
unconditional probability of moving (1.32\%). The results indicate
that there is a pronounced age effect, with each additional year of
age implying a reduction that is equal to 6\% of the baseline probability.
The same effect is found for whether or not children are present in
the household. The effect of being a homeowner is very large and equivalent
to a reduction in the propensity to move of 51\% of the baseline probability.
Increasing household income by \$100,000 is equivalent to a 5\% baseline
increase. Finally, having a college degree has an effect of equal
magnitude than being a homeowner, but in the opposite direction: a
college degree amounts to an increase of the baseline of 49\%. According
to this model, the effect of being a homeowner on the baseline moving
probability is equal to an age increase of 8.3 years, thus taking
a 30-year old to age 38; also, a household which owns the house would
have to experience an increase in household income of \$1m in order
to make up for the implied loss in the probability of moving across
divisions from being an owner. The house price to income ratio and
total household wealth are not statistically significant in this specification.

\paragraph{Sample Selection: Non\textendash College Degree }

Even though the estimates in table \ref{tab:D2D-probit} only measure
statistical associations, they highlight an important feature of the
data: moving and having a college degree are strongly correlated.
While this paper specifically aims to investigate the other strong
correlation in that table, i.e. between ownership status and mobility,
a full treatment which endogenizes education choices is too ambitious.
A pragmatic solution to this problem is to condition the data on a
certain education group and disregard education choices, as is done
in the previous literature (e.g. \cite{Bishop:2008fk,ransom2018labor,bartik2018moving,kennan2003effect}
all impose this restriction). In what follows, therefore, all SIPP
data will refer to household heads without a college degree, which
selects 62\% of the original sample, resulting in 65,482 unique household
heads.

\subsection{Renters Move at Twice the Rate of Owners at all Ages}

In order to give a sense of the magnitude of migration rates by ownership
status in this selected sample, table \ref{tab:Annual-moving-rate}
presents summary annual moving rates for both State and Census Division
level migration. The overall unconditional migration rate is 1.51\%
and 0.99\% of households per year for cross State and cross Division,
respectively. The cross State figure differs from the 2\% in table
\ref{tab:Migration-rates-Molloy} because I set up the SIPP data in
terms of household heads, thereby missing some moves of non\textendash reference
persons, and because I condition on non\textendash college. It is
quite clear from table \ref{tab:Annual-moving-rate} that there is
a marked distinction in the likelihood of moving across State as well
as Division boundaries between renters and owners, with 2.07\% (1.49\%)
of renters versus 0.82\% (0.64\%) of owners moving across State (Division)
boundaries on average per year. In total I observe 1259 cross Division
moves made by 1069 unique individuals in my non-college sample, implying
multiple moves for some movers.\footnote{By way of comparison, the estimation sample in \cite{kennan2003effect}
is drawn from the geo-coded version of NLSY79 and contains 124 interstate
moves. The disadvantage of SIPP is I can track an individual for at
most four years.} 

Reconsidering homeowership and migration by age gives rise to figure
\ref{fig:SIPP-sample-proportion-1}. It is clear that renters are
more likely to move at all ages, with a strongly declining age effect
\textendash{} younger individuals move more. At the same time, homeownership
is increasing with age. These are highly salient features of the data,
and they are among the key dimensions along which this model's performance
is going to be evaluated. 

\subsection{Regional Income and House Price Risk are not IID\label{subsec:Income-not-IID}}

The time series of regional disposable income and regional house prices
are each strongly correlated across Divisions. Additionally, they
exhibit high degress of autocorrelation, i.e. shocks to regional incomes
and prices are persistent. To illustrate the degree of cross correlation
of both prices and incomes consider figure \ref{fig:TS-correlogram}.
The top panels show the detrended version of each time series, by
region, while the bottom panels show the pairwise correlation of those
detrended time series across regions. The figure highlights that deviations
from trend are highly correlated between Divisions, for both average
regional incomes $(q)$ and regional prices $(p)$.\footnote{Data for $q$ comes from the BEA series ``Personal Income by State'',
$p$ is the FHFA house price index by Census Division. Both sets of
series are a direct input to the structural model to be introduced
below. Data are available via \url{https://github.com/floswald/EconData}.} Regarding persistence of those time series, the average autorcorrelation
coefficients are 0.91 for $p$ and 0.92 for $q$, respectively (for
details see online appendix table \ref{R-tab:Autocorrs}) over the
considered time period. Modelling regional risk as an IID process
seems like an unjustifiably strong assumptions given those high degrees
of cross correlation and persistence. Therefore the model introduced
below will take both correlation and persistence in regional prices
seriously and will propose a method to solve and estimate the resulting
high-dimensional problem. Online appendix \ref{R-subsec:DimReduce-Data}
contains detailed descriptions of the raw data.

\begin{table}[H]
\begin{centering}
\input{\Dropbox/data/cps/main-reason.tex}
\par\end{centering}
\caption{\label{tab:CPS-reason-move}CPS 2013 data on main motivation of moving,
conditional on a cross Division move. The purpose of this table is
to show that the distribution of responses is stable conditional on
quartiles of \emph{distance moved.} This selects a sample of 20-50
year-olds and aggregates the response to the question ``What was
the main reason for moving'' (variable NXTRES) as follows. Work =
\{new job/transfer, look for job, closer to work, retired\}, Housing
= \{estab. own household, want to own, better house, better neighborhood,
cheaper housing, foreclosure, other housing\}, family = \{change marstat,
other fam reason\}, other = \{attend/leave college, climate change,
health, natural disaster, other\}. The distance of a move is computed
as the distance between geographic center of the \emph{state} of origin
(not Division) and the center of the destination state. The rows of
the table categorize the distance measure into its quartiles.}
\end{table}

\begin{table}[H]
\begin{centering}
\input{\Dropbox/data/sipp/move_rates.tex}
\par\end{centering}
\caption{Annual moving rate in percent of the population. Households are categorized
into ``Renter'' or ``Owner'' based on their homeownership status
at the beginning of the period in which they move. SIPP data subset
to non-college degree holders. \label{tab:Annual-moving-rate}}
\end{table}

\begin{table}[H]
\begin{centering}
\input{\Dropbox/data/sipp/D2D-probit-edit.tex}
\par\end{centering}
\caption{Determinants of cross census division moves in SIPP data. Household
income and wealth are measured in 100,000 USD. This regresses a binary
indicator for whether a cross division move takes place at age $t$
on a set of variables relevant at that date. The first column shows
marginal effects, the second column shows the marginal effects relative
to the unconditional baseline mobility rate of 0.0132. The interpretation
of this column is for example that the effect of being a homeowner
is equivalent to reducing the baseline probability of migration by
51\%.\label{tab:D2D-probit}}
\end{table}

\begin{figure}
\begin{centering}
\includegraphics[scale=0.85]{\Dropbox/data/sipp/raw-moversD2D_v2}
\par\end{centering}
\caption{SIPP sample proportion moving across Census Division boundaries by
age (upper panel) and proportion of owners by age (lower panel). Conditions
on individuals without a college degree.\label{fig:SIPP-sample-proportion-1}}
\end{figure}

\begin{figure}
\begin{adjustbox}{center}
\begin{centering}
\includegraphics[scale=0.9]{\Dropbox/data/FHFA/correlogram-detrended}
\par\end{centering}
\end{adjustbox}

\caption{The time series for regional incomes $\{q_{dt}\}_{t=1967}^{2012}$
and house prices $\{p_{dt}\}_{t=1967}^{2012}$ are strongly correlated
across Divisions $d$. The top panels of this figure show the detrended
series for all Divisions for both $q$ and $p$. The bottom panel
further emphasizes that the cross-correlations across states between
regional trend deviations are substantial. For instance in the bottom
left panel, the top left tile indicates that the correlation between
the time series of $q$ for East North Central (ENC) and West South
Central (WSC) is around 0.75 (raw numbers in appendix tables \ref{R-tab:Cross-corr-q}
and \ref{R-tab:Cross-corr-p}). Detrended with fourth-order moving
average.\label{fig:TS-correlogram}}

\end{figure}

\newpage

\section{Model\label{sec:Model}}

In the model I view households as a single unit, and I'll use the
terms \emph{household }and \emph{individual} interchangeably. Individuals
are assumed to live in census Division (or \emph{region}) $d\in D$
in any given period at date $t$, and we let $j\in\left\{ 1,\dots,J\right\} $
index age. At each age $j$, individual $i$ has to decide whether
to move to a different region $k$, whether to own or rent, and how
much of his labor income to save. Individuals derive utility from
consumption $c$, from owning a house $h$ and from local amenity
$A_{d}$.

Every individual in region $d$ faces an identical level of house
price $p_{dt}$ and mean labor productivity $q_{dt}$ at time $t$,
where $q_{dt}$ enters the individual wage equation as a level shifter.
At the individual level uncertainty enters the model through a Markovian
idiosyncratic component of income risk $z_{ij}$, a Markovian process
that models changes in household size over the lifecycle $s_{ij}$,
and a location\textendash specific preference shock $\varepsilon_{idt}$,
which is assumed identically and independently distributed across
agents, regions and time. In short, region $d$ is characterized by
a tuple $(q_{dt},p_{dt},A_{d})$, households can move to a different
region subject to a moving cost, and they hold expectations about
the evolution of regional prices $\left(q_{dt+1},p_{dt+1}\right),\forall d$
in such a way that is compatible with the evidence from figure \ref{fig:TS-correlogram}
(i.e. correlated shocks across region and high persistence) and is
at the same time computationally feasible, as detailed below.\footnote{Let it suffice for now to state that taking into account 9 different
house price and labour income processes would not be feasible, and
therefore the solution will seek to reduce the number of relevant
dimensions of these series, similar to what a principal component
analysis would try to do.}

The job search process is modeled as in \cite{kennan2003effect}.
Individuals do not know the exact wage they will earn in the new location.
The new wage is composed of a deterministic, and thus predictable,
part and a component that is random. Over and above an expectation
about some prevailing average level of wages the mover can expect
in any given region at time $t$, it is impossible to be certain about
the exact match quality of the new job ex ante. The new job can be
viewed as an experience good where quality is revealed only after
an initial period. This setup gives rise to income risk associated
with moving. I do not attempt to explain return migrations, which
\cite{kennan2003effect} achieve with a region-person specific match
effect and by including this match effect from the last location in
the state space. \footnote{Adding this feature would increase the computational burden of the
model to make it infeasible, even with the limited memory assumption
employed in \cite{kennan2003effect}. I do not expect return migration
to be of first order for the questions adressed here.} 

The model describes the partial equilibrium response of workers to
regional wage and price shocks, as well as idiosyncratic income and
family size shocks. The fairly detailed description of the consumer's
decision problem rules out a full equilibrium analysis where house
prices and wages clear local markets for computational feasibility
reasons, hence, $(q_{dt},p_{dt})$ are exogenous to the model.

\subsection{Individual Labor Income\label{subsec:Individual-Labor-Income}}

The logarithm of labor income of individual $i$ depends on age $j$,
time $t$, and current region $d$ and is defined as in equation \eqref{eq:income-process}.
\begin{eqnarray}
\ln y_{ijdt} & = & \eta_{d}\ln q_{dt}+f(j)+z_{it}\nonumber \\
z_{it} & = & \rho z_{it-1}+e_{it-1}\nonumber \\
e & \sim & N(0,\sigma^{2})\label{eq:income-process}
\end{eqnarray}
Here $q_{dt}$ stands for the region specific price of human capital,
$f(j)$ is a deterministic age effect modeled as a nonlinear function
and $z_{it}$ is an individual specific persistent idiosyncratic shock.
The coefficient $\eta_{d}$ allows for differential transmission of
regional shocks into individual income by region $d$. The log price
of human capital $q_{dt}$ is allowed to differ by region to reflect
different industry compositions by region, which are taken as given.\footnote{ Underlying this is an assumption about non\textendash equalizing
factor prices across regions. It is plausible to think that within
a single country, wages should tend to converge to a common level,
particularly in the presence of large migratory flows from one region
to the next. In assuming no relative factor price equalization across
US regions I rely on a host of evidence showing that relative wages
vary considerably across regions over a long time horizon (see for
example \cite{redding}). }

When moving from region $d$ to region $k$ at date $t$, I assume
that the timing is such that current period income is earned in the
origin location $d$. The individual's next period income is then
composed of the corresponding mean income at that date in the new
region $k$, $q_{kt+1}$, the deterministic age $j+1$ effect, $f(j+1)$,
and a new draw for $z_{it+1}$ conditional on their current shock
$z_{it}$. For a mover, this individual\textendash specific idiosyncratic
component is drawn from a different conditional distribution than
for non-movers. Let us denote the different conditional distributions
of $z_{it+1}$ given $z_{it}$ for stayers and movers by $G_{\text{stay}}$
and $G_{\text{move}}$, respectively. This setup allows for some uncertainty
related to the quality of the match with a job in the new region $k$,
as mentioned above. In the model I use $G_{\text{stay}}$ and $G_{\text{move}}$
as transition matrices from state $z$ today to state $z'$ tomorrow
for stayers and movers, respectively. 

\subsection{Dimensionality Reduction: National factors $P$ and $Q$\label{subsec:Dim-Reduce}}

As stated above, allowing $\left(q_{dt+1},p_{dt+1}\right),\forall d$
to vary in an unrestricted fashion would make computation of this
model infeasible. To solve this problem, I assume that agents use
a 2-dimensional factor model to infer regional prices.\footnote{The method of \cite{krusell_smith} is conceptually similar to what
I'm doing. Instead of mean and variance of a distribution, consumers
here track the value of two aggregate state variables.} To this end I define aggregate state variables $Q$ and $P$, which
evolve according to a stationary vector autoregression of order one.
At date $t$, all individuals observe the price vector $\mathbf{F}_{t}$
containing both $P_{t}$ and $Q_{t}$. The process is formally defined
in equation \eqref{eq:VAR-process-agg}, where $A$ is a matrix of
coefficients and $\Sigma$ is the variance-covariance matrix of the
bivariate normal innovation $\nu$. Agents in the model have rational
expectations concerning this process.
\begin{eqnarray}
\mathbf{F}_{t} & = & A\mathbf{F}_{t-1}+\nu_{t-1}\nonumber \\
\nu_{t} & \sim & N\left(\begin{bmatrix}0\\
0
\end{bmatrix},\Sigma\right)\label{eq:VAR-process-agg}\\
\mathbf{F}_{t} & = & \begin{bmatrix}Q_{t}\\
P_{t}
\end{bmatrix}\nonumber 
\end{eqnarray}


\subsubsection{Mapping aggregate factors to regional prices}

I assume that there is a deterministic mapping from the aggregate
state $\mathbf{F}_{t}$ into the price and income level of any region
$d$ which is known by all agents in the model. This means that once
the aggregate state is known, agents know the price $p_{dt}$ and
income level $q_{dt}$ in each region with certainty. The mapping
is defined in terms of a function that depends on both aggregate states
$Q,P$ and where the coefficients are region dependent, as shown in
expression \eqref{eq:VAR-reg-determ}. Similarly to the aggregate
case in \eqref{eq:VAR-process-agg}, $\mathbf{a}_{d}$ is a 2x2 matrix
of coefficients specific to region $d$.

\begin{eqnarray}
\begin{bmatrix}q_{dt}\\
p_{dt}
\end{bmatrix} & = & \mathbf{a}_{d}\mathbf{F}_{t}\label{eq:VAR-reg-determ}
\end{eqnarray}

Notice that the great virtue of this formulation is that the relevant
price and income related state variables in each region are subsumed
in $\mathbf{F}_{t}$, given the assumption that $\mathbf{a}_{d}$
is known for all $d$. To be completely clear, equation \eqref{eq:VAR-reg-determ}
shuts down any uncertainty at the regional level once $\mathbf{F}_{t}$
and $\mathbf{a}_{d}$ are known.\footnote{One could say that the formulation is missing a shock, e.g. $\mathbf{a}_{d}\mathbf{F}_{t}+\epsilon_{dt},\epsilon_{dt}\sim N(0,\sigma^{2})$.
Adding such a shock would increase the state space by a factor equal
to the number of integration nodes to be used for the approximation
of the resulting integral, which is a big cost. I do not expect any
major difference in my results. The fit of this approximation is very
good, as will be shown further below.} Shocks materialize in region $d$ as a transformation of aggregate
shocks to $Q$ and $P$. The implications of this will be discussed
in greater detail in section \ref{subsec:Aggregate-to-regional} when
I describe estimation of this part of the model and where I also provide
some illustration regarding the fit of this model to the data.

\subsection{Home Ownership Choice\label{subsec:Home-Ownership-Choice}}

Ownership choice is discrete, $h_{j}\in\left\{ 0,1\right\} $, and
there is no quantity choice of housing. While renting, i.e. whenever
$h_{j}=0$, individuals must pay rent which amounts to a constant
fraction $\kappa_{d}$ of the current region-$d$ house price $p_{d}$.
Similar to the setup in \cite{Attanasio20121}, I denote total financial
(i.e. non-housing) wealth at age $j$ as assets $a_{j}$, which
include liquid savings and mortgage debt. There is a terminal condition
for net wealth to be non-negative by the last period of life, i.e.
$a_{J}+p_{dt}h_{J-1}\geq0,\forall t$, which translates into an implicit
borrowing limit for owners. Additional to that, in order to buy, a
proportion $\chi p_{dt}$ of the house value needs to be paid up front
as a downpayment, while the remainder $(1-\chi)p_{dt}$ is financed
by a standard fixed rate mortgage with exogenous interest rate $r^{m}$.
The mortgage interest rate is assumed at a constant markup $\hat{r}>0$
above the risk free interest rate $r$, such that $r^{m}=r+\hat{r}$.
The markup captures default risk incurred by a mortgage lender. 

The equity constraint must be satisfied in each period, i.e. $a_{ij+1}\geq-(1-\chi)p_{dt}h_{j},\forall t$.
This means that only owners are allowed to borrow, with their house
as a collateral. Selling the house incurs proportional transaction
cost $\phi$, such that given current house price $p_{t}$, upon sale
the owner receives $(1-\phi)p_{t}$.

This setup implies that owners will choose a savings path contingent
on the current price, their income and debt level, the mortgage interest
rate, and their current age $j$, such that they can satisfy the final
period constraint. Subsections \ref{subsec:Budget-stayers} and \ref{subsec:Budget-movers}
below describe the budget constraints in greater detail.

\subsection{Moving\label{subsec:Moving}}

\paragraph{Moving Costs.}

Moving is costly both in monetary terms (see the budget constraints
below in \ref{subsec:Recursive-Formulation}) and in terms of utility.
Denote $\Delta(d,x)$ the utility costs of moving from $d$ at a current
value of the state vector $x$ (defined below). Moving costs differ
between renters and owners. Moving for an owner requires to sell the
house, which in turn requires some effort and time costs. This is
in addition to any other utility costs incurred from moving regions
which are common between renters and owners. I specify the moving
cost function as linear in parameters $\alpha$:
\begin{equation}
\Delta(d,x)=\alpha_{0,\tau}+\alpha_{1}j+\alpha_{2}j^{2}+\alpha_{3}h_{ij-1}+\alpha_{4}s_{ij}\label{eq:moving-cost}
\end{equation}
In expression \eqref{eq:moving-cost}, $\alpha_{0,\tau}$ is an intercept
that varies by unobserved moving cost type $\tau$, $\alpha_{1}$
and $\alpha_{2}$ are age effects, $\alpha_{3}$ measures the additional
moving cost for owners, and $\alpha_{4}$ measures moving cost differential
arising from family size $s_{ij}$.

The unobserved moving cost type $\tau\in\{0,1\}$, where $\tau=1$
indexes the high\textendash cost type, is a parsimonious way to account
for the fact that in the data, some individuals never move. This is
of particular relevance when thinking about owners, who may self\textendash select
into ownership because they know they are unlikely to ever move. In
the model this selection mechanism, together with any other factor
that implies a high unobservable location preference, is collapsed
into a type of person that has prohibitively high moving costs ($\alpha_{0,\tau=1}$
is large) and thus is unlikely to move. Providing some real-world
context for this setup, \cite{kocsar2019understanding} use consumer
expectations data to find that for close to 50\% of the population,
non-pecuniary moving costs approach Infinity.

\paragraph{Restrictions.}

I rule out the possibility of owning a home in region $d$ while residing
in region $k$. This would apply for example for households who keep
their home in $d$, rent it out on the rental market, and purchase
housing services either in rental or owner\textendash occupied sector
in the new region $k$. In my sample I observe less than 1\% of movers
for which this is the case. Most likely this is a result of high
managment fees or a binding liquidity constraint that forces households
to sell the house to be able to afford the downpayment in the new
region.\footnote{SIPP allows me to verify whether individuals possess any real estate
other than their current home at any point in time. Fewer than 1\%
of movers provide an affirmative answer to this.} 

\subsection{Preferences}

Period utility $u$ depends on the choice of region $k$,
and whether this is different from the current region $d$. A move
takes place in the former case, and the household stays in $d$ in
the latter case. 
\begin{equation}
u\left(c,h,k;x_{it}\right)=\eta\frac{c^{1-\gamma}}{1-\gamma}+\xi(s_{ij})\times h-\mathbf{1}\left[d\neq k\right]\Delta\left(d',x_{it}\right)+A_{k}+\varepsilon_{ikt}\label{eq:utility}
\end{equation}
Notice that $\left(c,h,k\right)$ are \emph{current }period choices
of consumption, housing status and location that affect utility. Those
choices interact with the value of the state vector $x_{it}$, hence
they depend on household sizes $s_{ij}$, and an additively seperable
idiosyncratic preference shock for the chosen region $k$, $\varepsilon_{ikt}$.
Parameter $\eta$ measures the scale at which consumption enters utility,
while $\xi$ measures the importance of ownership at various household
sizes $s$. Household size $s$ at age $j$ is a binary random variable,
$s\in\left\{ 0,1\right\} $, relating to whether or not children are
present in the household. It evolves from one period to the next in
an age-dependent way as described in section \ref{subsec:Recursive-Formulation}.
Moving costs $\Delta\left(k,x_{it}\right)$ are only incurred if in
fact a move takes place. Finally, amenities in region $k$ are given
by the fixed effect $A_{k}$.

\subsection{Timing and State Vector}

The state vector of individual $i$ at date $t$ when they are of
age $j$ is given by 
\[
x_{it}=\left(a_{ij},z_{ij},s_{ij},\mathbf{F}_{t},h_{ij-1},d,\tau,j\right)
\]
where the variables stand for, in order, assets, individual income
shock, household size, aggregate price vector, housing status coming
into the current period, current region index, moving cost type and
age.\footnote{A word of caution regarding the two time indices $j$ and $t$: For
large parts of the exposition this distinction is irrelevant, i.e.
saying $a_{ij}$ or $a_{it}$ is equivalent. However, in the estimation
I will allow different cohorts $C_{1},\dots,C_{N}$ to experience
different sequences of prices $\mathbf{F}_{C_{1}},\mathbf{F}_{C_{2}},\dots$,
and therefore separating time and age will become necessary.}

Timing within the period is assumed to proceed in two sub-periods:
in the first part, stochastic states are realized and observed by
the agent, and labor income is earned; in the second part the agent
makes optimal decisions regarding consumption, housing and location.
The chronological order within a period is thus as follows:
\begin{enumerate}
\item observe $\mathbf{F}_{t}$, $s_{it},z_{ij}$ and $\varepsilon_{it}=\left(\varepsilon_{i1t},\varepsilon_{i2t},\dots,\varepsilon_{iDt}\right)$,
iid location taste shock 
\item earn labor income in current region $d$, as a function of $q_{dt}$
and $z_{ij}$
\item given the state, compute optimal behaviour in all $D$ regions, i.e.
\begin{enumerate}
\item choose optimal consumption $c_{h}^{*}$ conditional on housing choice
$h\in\{0,1\}$ in all regions $k$
\item choose optimal housing $h_{d}^{*}(c_{h}^{*})$
\item choose optimal location, based on the value of optimal housing in
each location
\end{enumerate}
\end{enumerate}

\subsection{Recursive Formulation\label{subsec:Recursive-Formulation}}

It is now possible to formulate the problem recursively. Following
\cite{rust-bus}, I have assumed additive separability between utility
and idiosyncratic location shock $\varepsilon$ as well as independence
of the transition of $\varepsilon$ conditional on $x$. Furthermore,
I assume that $\varepsilon$ is distributed according to the Standard
Type 1 Extreme Value distribution.\footnote{This is also called the Standard Gumbel distribution. Notice that
the \emph{Standard} part implies that location and scale parameters
of the Gumbel distribution are chosen such that $E[\varepsilon]=\bar{\gamma}$,
a constant known as the Euler-Mascheroni number, and that the its
standard deviation is fixed at $\sqrt{Var(\varepsilon)=}\frac{\pi}{\sqrt{6}}$.}

The consumer faces a nested optimization problem in each period. At
the lower level, optimal savings and housing decision must be taken
conditional on any discrete location choice, and at the upper level
the discrete location choice with the maximal value is chosen, see
\eqref{eq:V}. It is useful to define the conditional value function
$v\left(x,k\right)$, which represents the optimal value after making
housing and consumption choices at state $x$, while moving to location
$k$, net of idiosyncratic location shock $\varepsilon$, in \eqref{eq:v}.
Equation \eqref{eq:vbar} is a result of the distributional assumption
on $\varepsilon$, which admits a closed form expression of the expected
value function (also known as the \emph{Emax }function in this model
class), whereby $\bar{\gamma}\approx0.577$ is Euler's constant. 

\begin{eqnarray}
V\left(x_{it}\right) & = & \max_{k\in D}\left\{ v\left(x_{it},k\right)\right\} \label{eq:V}\\
v\left(x_{it},k\right) & = & \max_{c>0,h\in\left\{ 0,1\right\} }u\left(c,h,k;x_{it}\right)+\varepsilon_{ikt}+\beta\mathbb{E}_{z,s,\mathbf{F}}\left[\overline{v}\left(x_{it+1}\right)|z_{ij},s_{ij},\mathbf{F}_{t}\right]\label{eq:v}\\
x_{it+1} & = & \left(a_{ij+1},z_{ij+1},s_{ij+1},\mathbf{F}_{t+1},h,k,j+1\right)\nonumber \\
\overline{v}\left(x_{it+1}\right) & = & E_{\varepsilon}V\left(x_{it+1}\right)\nonumber \\
 & = & \overline{\gamma}+\ln\left(\sum_{k=1}^{D}\exp\left(v\left(x_{it+1},k\right)\right)\right)\label{eq:vbar}
\end{eqnarray}

The final period models a terminal value that depends on net wealth
and a term that captures future utility from the house after age $J$,
as shown in equation \eqref{eq:final-period-value}.
\begin{equation}
V_{J}(a,h_{J-1},d)=\frac{\left(a_{J}+h_{J-1}p_{dt}\right)^{1-\gamma}}{1-\gamma}+\omega h_{J-1},\forall t\label{eq:final-period-value}
\end{equation}
The maximization problem in equation \eqref{eq:v} is subject to several
constraints, which vary by housing status and location choice. It
is convenient to lay them out here case by case.

\subsubsection{Budget constraint for stayers, i.e. $d=k$\label{subsec:Budget-stayers}}

Starting with the case for stayers, the relevant state variables in
the budget constraint refer only to the current region $d$. In particular,
given $(p_{dt},q_{dt})$, renters may choose to become owners, and
owners may choose to remain owners or sell the house and rent.

\paragraph{Renters.}

The period budget constraint for renters (i.e. individuals who enter
the period with $h_{ij-1}=0$) depends on their housing choice, as
shown in equation \eqref{eq:budget-renter-stay}. In case they buy
at date $t$, i.e. $h_{ij}=1$, they need to pay the date $t$ house
price in region $d$, $p_{dt}$, otherwise they need to pay the current
local rent, $\kappa_{d}p_{dt}$. Labor income is defined in equation
\eqref{eq:budget-labor-income} and depends on the regional mean labor
productivity level $q_{dt}$ as introduced in section \ref{subsec:Individual-Labor-Income}.
Buyers can borrow against the value of their house and are required
to make a proportional downpayment amounting to a fraction $\chi$
of the value at purchase, while renters cannot borrow at all. This
is embedded in constraint \eqref{eq:budget-rent-downpay}, which states
that if a renter chooses to buy, their next period assets must be
greater or equal to the fraction of the purchase price that was financed
via the mortgage, or non-negative otherwise. Constraint number \eqref{eq:budget-interest}
defines the interest rate function, which simply states that there
is a different interest applicable to savings as opposed to borrowing,
both of which are taken as exogenous parameters in the model. $\hat{r}$
stands for the exogenous risk premium of mortages charged over the
risk free rate. The terminal condition constraint is in expression
\eqref{eq:terminal-condition}.

\begin{eqnarray}
a_{ij+1} & = & (1+r(a_{ij}))\left(a_{ij}+y_{ijdt}-c_{ij}-(1-h_{ij})\kappa_{d}p_{dt}-h_{ij}p_{dt}\right)\label{eq:budget-renter-stay}\\
\ln y_{ijdt} & = & \eta_{d}\ln(q_{dt})+f(j)+z_{ij}\label{eq:budget-labor-income}\\
a_{ij+1} & \geq & -(1-\chi)p_{dt}h_{ij}\label{eq:budget-rent-downpay}\\
r(a_{ij}) & = & \begin{cases}
r & \text{if }a_{ij}\geq0\\
r^{m} & \text{if }a_{ij}<0
\end{cases},r^{m}=r+\hat{r}\label{eq:budget-interest}\\
a_{iJ}+p_{t}h_{iJ-1} & \geq & 0,\forall t\label{eq:terminal-condition}
\end{eqnarray}


\paragraph{Owners.}

For individuals entering the period as owners $\left(h_{ij-1}=1\right)$,
the budget constraint is similar except for two differences which
relate to the borrowing constraint and transfers in case they sell
the house. Owners are not required to make a scheduled mortgage payment
\textendash{} a gradual reduction of debt, i.e. an increase in $a$,
arises naturally from the terminal condition $a_{iJ}+p_{t}h_{iJ-1}\geq0,\forall t$,
as mentioned above. Therefore the budget of the owner is only affected
by the house price in case they decide to sell the house, i.e. if
$h_{ij}=0$. In this case, they obtain the house price net of the
proportional selling cost $\phi$, plus they have to pay rent in region
$d$. Apart from this, the same interest rate function \eqref{eq:budget-interest},
labor income equation \eqref{eq:budget-labor-income} and terminal
condition \eqref{eq:terminal-condition} apply. 
\begin{eqnarray}
a_{ij+1} & = & (1+r(a_{ij}))\left(a_{ij}+y_{ijdt}-c_{ij}+(1-h_{ij})(1-\phi-\kappa_{d})p_{dt}\right)\label{eq:budget-own}\\
a_{ij+1} & \geq & -(1-\chi)p_{dt}\label{eq:budget-own-borrow}
\end{eqnarray}


\subsubsection{Budget constraint for movers, i.e. $d\protect\neq k$\label{subsec:Budget-movers}}

\paragraph{Renters.}

For moving renters the budget constraint is close to identical, with
the exception that \eqref{eq:budget-renter-stay} needs to be slightly
altered to reflect that labor income is obtained in the current period
in region $d$ before the move to $k$ is undertaken. 
\begin{equation}
a_{ij+1}=(1+r(a_{ij}))\left(a_{ij}+y_{ijdt}-c_{ij}-(1-h_{ij})\kappa_{k}p_{jt}-h_{ij}p_{kt}\right)\label{eq:budget-rent-move}
\end{equation}


\paragraph{Owners.}

The budget constraint for moving owners depends on the house price
in both current and destination regions $d$ and $k$ since the house
in the current region must be sold by assumption. The expression $(1-\phi)p_{dt}$
in \eqref{eq:budget-own-1} relates to proceeds from sale of the house
in region $d$, whereas the square brackets describe expenditures
in region $k$. Notice also that the borrowing constraint \eqref{eq:budget-own-borrow-1}
now is a function of the value of the new house in $k$. It is important
to note that this formulation precludes moving with negative equity
if labor income is not enough to cover it. This is exacerbated in
cases where the mover wants to buy immediately in the new region,
since in that case the downpayment needs to be made as well, i.e.
if $y_{ijdt}<a_{ij}+(1-\phi)p_{dt}-\chi h_{ij}p_{kt}$ then the budget
set is empty and moving and buying is infeasible.\footnote{In my sample I observe 29 owners who move with negative equity (amounting
to 3.4\% of moving owners). 78\% of those do buy in the new location,
the rest rent. I do not observe whether or not an owner defaults on
the mortgage. Accounting for this subset of the population would require
to 1) assume that they actually defaulted and 2) it would substantially
increase the computational burden. For those reasons the model cannot
account for this subset of the mover population at the moment.} 

\begin{eqnarray}
a_{ij+1} & = & (1+r(a_{ij}))\left(a_{ij}+y_{ijdt}-c_{ij}+(1-\phi)p_{dt}-\left[(1-h_{ij})\kappa_{k}+h_{ij}\right]p_{kt}\right)\label{eq:budget-own-1}\\
a_{ij+1} & \geq & -(1-\chi)p_{kt}h_{ij}\label{eq:budget-own-borrow-1}
\end{eqnarray}


\section{Solving and Simulating the Model \label{sec:Solving-and-Simulating}}

The model described above is a typical application of a mixed discrete\textendash continuous
choice problem. In the next section I will introduce a nested fixed
point estimator, which requires repeated evaluation of the model solution
at each parameter guess, thus placing a binding time-contraint on
time each solution may take.

The consumption/savings problem to be solved at each state, and its
combination with multiple discrete choices and borrowing constraints,
introduces several non-differentiabilities in the asset dimension
of the value function. This makes using fast first order condition\textendash based
approaches to solve the consumption problem more difficult.\footnote{There has recently been a lot of progress on this front. \cite{clausenenvelope}
provide an envolope theorem for the current case, and the endogenous
grid point method developed by \cite{carroll2006method}, further
extended to accomodate (multiple) discrete choice as in \cite{fella2011generalized}
and \cite{iskhakov2017endogenous} are promising avenues. I did not
further pursue conditional choice probability (CCP) methods as in
\cite{arcidiacono2008ccp} or \cite{IERE:IERE12001}, for example,
because of data limitations. I experimented in particular with the
latter paper's approach but soon had to give up because of too many
empty cells in the empirical choice probability matrix (e.g. an entry
like $\Pr(\text{own},\text{save}=s,\text{move}|X)$ would be empty
for many values of $X$; in general, my problem was to recover the
first stage decision rules form the data in a satisfactory kind of
way).} 

I solve the model in a backward-recursive way, starting at maximal
age 50 and going back until initial age 20. In the final period the
known value is computed at all relevant states. From period $J-1$
onwards, the algorithm in each period iterates over all state variables
and computes a solution to the savings problem at each combination
of state and discrete choices variables (including housing and location
choices). Notice that this state space spans all values for $\mathbf{F}_{t}$
observed over the sample period. After this solution is obtained at
a certain state, the discrete housing choice is computed, after which
each conditional value function \eqref{eq:v} is known. 

Once the solution is obtained, simulation of the model proceeds by
using the model implied decision rules and the observed aggreate prices
series $\mathbf{F}_{t}$ as well as their regional dependants $\left(q_{dt},p_{dt}\right)$
to obtain simulated lifecycle data. As will become clear in the next
section, this procedure needs to replicate the time and age structure
found in the data, which is achieved by simulating different cohorts,
starting life in 1967 and all successive years up until 2012. The
model moments are then computed using the empirical age distribution
found in the estimation sample as sampling weights. 

\section{Estimation \label{sec:Estimation}}

In this section I explain how the model is estimated to fit some features
of the data. There is a set of preset model parameters, the values
of which I either take from other papers in the literature or I estimate
them outside of the structural model and treat them as inputs. The
remaining set of parameters are estimated using the simulated method
of moments (SMM) approach, whereby given a set of parameters, the
model is used to compute decision rules of agents, which in turn are
used to simulate artificial data. In what follows, I will first discuss
estimation of the exogenous stochastic processes, and then turn to
the estimation of the model preference parameters. 

\subsection{Estimation of Exogenous Processes}

\subsubsection*{VAR process for aggregates $Q_{t}$ and $P_{t}$}

The VAR processes at the aggregate and regional level are estimated
using a seemingly unrelated regression with two equations, one for
each factor $Q_{t}$ and $P_{t},t=1967,\dots,2012$. I use real GDP
per capita as a measure for $Q_{t}$, and the Federal Housing and
Finance Association (FHFA) US house price index for $P_{t}$. Given
that I am interested in the level of house prices (i.e. a measure
of house \emph{value}), I compute the average level of house prices
found in SIPP data for the year 2012 and then apply the FHFA index
backwards to construct the house value for each year.\footnote{The GDP series is as provided by the Bureau of Economic Analysis through
the FRED database. All non-SIPP data used in this paper are provided
in an R package at \url{https://github.com/floswald/EconData}, documenting
all sources and data-cleaning procedures.} 

I reproduce equation \eqref{eq:VAR-process-agg} here for ease of
reading:
\begin{eqnarray*}
\mathbf{F}_{t} & = & A\mathbf{F}_{t-1}+\nu_{t-1}\\
\nu_{t} & \sim & N\left(\begin{bmatrix}0\\
0
\end{bmatrix},\Sigma\right)\\
\mathbf{F}_{t} & = & \begin{bmatrix}Q_{t}\\
P_{t}
\end{bmatrix}
\end{eqnarray*}

The estimates from this equation are given in table \ref{tab:Estimates-VAR-agg} and are used by agents in the model to predict $\mathbf{F}_{t+1}$ given $\mathbf{F}_{t}$.

\begin{table}
\begin{centering}
\begin{centering}
\input{\Dropbox/data/sipp/VAR_agg-edit.tex}
\end{centering}
\par\end{centering}
\caption{Estimates for Aggregate VAR process. $\{P_{t},Q_{t}\}_{t=1967}^{2012}$
are time series for FHFA national house prices, and real GDP per capita,
respectively.\label{tab:Estimates-VAR-agg}}
\end{table}


\subsubsection*{Aggregate to regional price mappings\label{subsec:Aggregate-to-regional}}

The series for $q_{dt}$ is constructed as per capita personal income
by region, with a measure of personal income obtained from the Bureau
of Economic Analysis and population counts by state from intercensal
estimates from the census Bureau. The price series by region, $p_{dt}$,
comes from the same FHFA dataset as used above.

\begin{eqnarray}
\begin{bmatrix}q_{dt}\\
p_{dt}
\end{bmatrix} & = & \mathbf{a}_{d}\mathbf{F}_{t}+\eta_{dt}\nonumber \\
\eta_{dt} & \sim & N\left(\begin{bmatrix}0\\
0
\end{bmatrix},\Omega_{d}\right)\label{eq:VAR-process-1}
\end{eqnarray}

The performance of this model in terms of delivered predictions from
the aggregate state can be gauged visually in figures \ref{fig:mapping-P-p}
and \ref{R-fig:mapping-Y-y}
in the online appendix \ref{R-sec:DimReduce}, which also contains
the respective parameters estimates in table \ref{R-tab:Aggregate-to-Regional-VAR}.
It is important to understand the purpose of models \eqref{eq:VAR-process-agg}
and \eqref{eq:VAR-process-1}: I do not want to make statistical inference
based on the estimates from those models, which is something they
may be ill-suited for, given the nature of the data. I am purely interested
in their ability to replicate the observed regional prices, when fed
the observed aggregate series for the purpose of approximating the
evolution of the prices state space during simulation. In that regard,
and by looking at \ref{fig:mapping-P-p} and \ref{R-fig:mapping-Y-y},
I find they perform well.

A different concern that might arise from looking at the models in
\eqref{eq:VAR-process-1} is that it is unclear a priori how they
in fact transform aggregate shocks into regional counterparts, as
this of course depends on the value of the estimated parameters $\mathbf{a}_{d}$.
An illustration of this translation of shocks is shown in appendix
\ref{R-subsec:Transformation-of-Aggregate}.

Finally, a reasonable concern is how good an approximation of a more
fine-grained geography such as state\textendash level this setup based
on Divisions is. In order to shed some light on this, I run pooled
OLS on my entire prices dataset, where on the left hand side I have
the price index for state $s$ in period $t$, $p_{st}$, and as explanatory
variable the corresponding Census Division level index, $p_{dt}$.
In table \ref{tab:-region-state-rsquared} we see the $R^{2}$
measured from each regression, implying that the division index is
explains a very large fraction of state-level variation throughout. The full regression output is in appendix
\ref{R-subsec:State-Division}.

\begin{table}
\begin{centering}
\input{\Dropbox/data/FHFA/rsquareds.tex}
\par\end{centering}
\caption{$R^{2}$ from pooled OLS regression of state level indices $p_{st},q_{st}$
on corresponding Division level indices $p_{dt},q_{dt}$.\label{tab:-region-state-rsquared}}
\end{table}


\begin{figure}
\begin{centering}
\includegraphics[scale=0.8]{\Dropbox/data/sipp/VAR_reg_p}
\par\end{centering}
\caption{This figure shows the observed and predicted time series for mean
income by Census Division. The prediction is obtained from the VAR
model in \eqref{eq:VAR-reg-determ}, which relates the aggreate series
$\left\{ Q_{t},P_{t}\right\} _{t=1968}^{2012}$ to mean labor productivity
$\left\{ q_{dt}\right\} _{t=1968}^{2012}$ for each region $d$\emph{.
}Agents use this prediction in the model, i.e. from observing an aggregate
value $\mathbf{F}_{t}=\left(P_{t},Q_{t}\right)$ they infer a value
for $q_{dt}$\emph{ }for each region above. \emph{\label{fig:mapping-P-p}}}
\end{figure}

\subsubsection*{Individual Income Process\label{subsec:Individual-Income-Process}}

This part deals with the empirical implementation of equation \eqref{eq:budget-labor-income},
which models log labor income at the individual level. I estimate
the linear regression 
\begin{eqnarray}
\ln y_{ijdt} & = & \beta_{0}+\eta_{d}\ln q_{dt}+\beta_{1}j_{it}+\beta_{2}j_{it}^{2}+\beta_{3}j_{it}^{3}+\beta_{4}\text{college}_{it}+z_{it}\label{eq:indiv-labor-empirical}
\end{eqnarray}
where $\text{college}_{it}=1$ if $i$ has a college degree, zero
else, and where $z_{it}$ are the regression residuals. The results
of this are shown in the appendix in table \ref{R-tab:Regional-2-indiv-income}
and figure \ref{R-fig:age-profiles}. The estimated residuals are
used together with parameters $\beta$ to generate an income grid
for individuals without college degree.

\subsubsection*{Copula estimates for $z$ Transistions \label{subsec:copula-estimates}}

The conditional distribution of $z$ for movers is specified as the
density of a bivariate normal copula $G_{\text{move}}$, which is
invariant to date and region.\footnote{A copula is a multivariate probability distribution function which
connects univariate margins by taking into account the underlying
dependence structure. For example, a finite state Markov transition
matrix is a nonparametric approximation to a bivariate copula, and
they converge as the number of states goes to infinity, see \cite{bonhomme2006modeling}.} This means I assume that the conditional probability of drawing $z'$
in new region $k$ is the same regardless the origin location.\footnote{It would be straightforward to relax this assumption, but data limitations
forced me to impose this restriction. }A copula is a multivariate distribution function with marginals that
are all uniformly distributed on the unit interval. For example, if
$F$ is a bi-dimensional CDF, and if $F_{i}$ is the CDF of the $i$-th
margin, then the bivariate copula is given by
\[
C(u_{1},u_{2})=F\left(F_{1}^{-1}(u_{1}),F_{2}^{-1}(u_{2})\right)
\]
where $F_{i}^{-1}$ is the quantile function. There are different
families of copulae, and I will use a normal copula. 

To estimate the parameters of the copula, I take residuals $z_{it}$
from equation \eqref{eq:indiv-labor-empirical} and I want to study
their joint distribution for movers, i.e. $\left(z_{it},z_{it+1}\right)|d\neq k$.
This object is informative for the question of whether individuals
with a particularly high residual $z_{it}$ are likely to have a high
residual $z_{it+1}$ after their move to region $k$, or not. In other
words, we want to investigate the joint distribution of stayers $\left(z_{it},z_{it+1}\right)|d=k$
and of movers $\left(z_{it},z_{it+1}\right)|d\neq k$ separately.
I obtain an estimate for the copula parameter $\rho_{s}$ of 0.58,
indicating substantial positive dependence for mover's $z$\footnote{$\rho_{s}$ is also called \emph{Spearman's rho,} and it is related
to Pearson's correlation coefficient $\rho_{p}$ via $2\sin\left(\frac{\pi}{6}\rho_{s}\right)=2\sin\left(\frac{\pi}{6}0.58\right)=\rho_{p}=0.598$
in this case of a gaussian copula. In particular, $\rho_{s}\in[-1,1]$.}. I report estimates and describe the full procedure in online appendix \ref{R-sec:Copula-Estimation}.
The conditional distribution of $z$ for non-movers will be parameterized
externally as explained next.

\subsubsection*{Values for preset parameters}

I take several parameters for the model from the literature, as shown
in table \ref{tab:Preset-parameter-values}. The estimates for the
components of the idiosyncratic income shock process for non-movers,
i.e. the autocorrelation $\rho=0.96$ and standard deviation of the
innovation $\sigma=0.118$ are taken from \cite{french_2005}. I set
the financial transaction cost of selling a house, $\phi$, to 6\%
in line with \cite{Li} and conventionally charged brokerage fees.
The time discount factor $\beta$ is set to 0.96 which lies within
the range of values commonly assumed in dynamic discrete choice models
(e.g. \cite{rust-bus}). The downpayment fraction $\chi$ is set to
20\%, which is a standard value on fixed rate mortgages and used throughout
the literature. The coefficient of relative risk aversion could be
estimated, but is in this version of the model fixed to 1.43 as in
\cite{attanasio_weber}.

To calibrate the interest rate for savings and for mortage debt, I
follow \cite{sommer2013implications}, who use the constant maturity
Federal Funds rate, adjusted by headline inflation as mesured by the
year on year change in the CPI. They obtain an average value of 4\%
for the period of 1977\textendash 2008, and I thus set $r=0.04$.
For the markup $q$ of mortgage interest over the risk-free rate they
use the average spread between nominal interest on a thirty year constant
maturity Treasury bond and the average nominal interest rate on 30
year mortgages. This spread equals 1.5\% over 1977\textendash 2008,
therefore $\hat{r}=0.015$, and $r^{m}=0.055$.

\begin{table}
\begin{centering}
\input{\Dropbox/model/fit/preset.tex}
\par\end{centering}
\caption{Preset parameter values\label{tab:Preset-parameter-values}}
\end{table}


\subsection{Estimation of Preference Parameters\label{subsec:Estimation-of-Preference}}

The parameter vector to be estimated by SMM contains the parameters
of the moving cost function ($\alpha$), the parameter in the final
period value function $\omega$, the population proportion of high
moving cost types $(\pi_{\tau})$, the scale of consumption $\eta$,
and the utility derived from housing for both household sizes, $\left(\xi_{1},\xi_{2}\right)$.
We denote the parameter vector of length $K$ as $\theta=\left\{ \alpha_{0},\alpha_{1},\alpha_{2},\alpha_{3},\alpha_{4},\omega,\pi_{\tau},\eta,\xi_{1},\xi_{2}\right\} $. 

Given $\theta$, the model generates a set of $M$ model moments $\hat{m}(\theta)\in\mathbb{R}^{M}$.
After obtaining the same set of moments $m$ from the data, the SMM
procedure seeks to minimize the criterion function 
\begin{equation}
L(\theta)=\left[m-\hat{m}(\theta)\right]^{T}W\left[m-\hat{m}(\theta)\right],\label{eq:objfunc}
\end{equation}
which delivers point estimate $\hat{\theta}=\arg\min_{\theta}L(\theta)$.
Given that this is a tightly parameterized model, I cannot use the
theoretically optimal weighting matrix $W$, because a range of economically
important moments vanish in the objective function because they enter
at different scales. This is equally true if I use the common strategy
of assigning the inverse of the variances of the data moments. To
solve this probem, I prespecify a $W$ as the identity matrix, but
I modify the diagonal entries for some moments so that the corresponding
derivative of the moment function is not negligible.\footnote{Notice that this procedure still leads to valid standard errors,
since $W$ appears together with the covariance matrix of moments
in the sandwich formula (see below). The weights are given by the
values 10 for moments \texttt{cov\_move\_h, mean\_move, mean\_move\_ownFALSE,
mean\_move\_ownTRUE} and \texttt{lm\_h\_age2}, 1.5 for all migratory
flow moments \texttt{flow\_move\_to\_j}, and finally by 2 for\texttt{
lm\_mv\_intercept} and\texttt{ cov\_own\_kids. }This adjustment is
similar to what is done in \cite{lamadon2014productivity}.}

The maximization of the objective in \eqref{eq:objfunc} is performed
with a cyclic coordinate search algorithm, where cycle $n+1$ is defined as follows:
\begin{align*}
\theta^{(n+1)}_1 & =\arg\min_{\theta_{1}} L(\theta_{1},\theta_{2}^{(n)},\dots,,\theta_{K}^{(n)})\\
\theta^{(n+1)}_2 & =\arg\min_{\theta_{2}} L(\theta_{1}^{(n)},\theta_{2},\theta_{3}^{(n)},\dots,\theta_{K}^{(n)})\\
\theta^{(n+1)}_3 & =\arg\min_{\theta_{3}} L(\theta_{1}^{(n)},\theta_{2}^{(n)},\theta_{3},\theta_{4}^{(n)},\dots,\theta_{K}^{(n)})\\
 & \vdots\\
\theta^{(n+1)}_K & =\arg\min_{\theta_{K}} L(\theta_{1}^{(n)},\theta_{2}^{(n)},\dots,\theta_{K}).
\end{align*}

This procedure is repeated until $\theta$ has converged. Convergence
was not affected by different starting values and occured in all cases
after less than 10 iterations over the above scheme.\footnote{This optimization takes around 16 hours on a 10-instance cluster on
AWS of type \texttt{t3.xlarge.} The procedure uses the function \texttt{optSlices}
in julia package \url{https://github.com/floswald/MomentOpt.jl}.}

Denoting $\theta_{0}$ the true parameter vector, by $\theta^{*}$
the optimizer of the above program and $\Sigma$ the variance-covariance
matrix of the asymptotic distribution of moment function errors as
in 
\[
\sqrt{n}(m-\hat{m}(\theta^{*}))\to\mathcal{N}(0,\Sigma),
\]

the distribution of the parameter estimates $\hat{\theta}$ is given
by the standard sandwich formula
\[
\sqrt{n}(\hat{\theta}-\theta_{0})\to\mathcal{N}\left(0,\left[dWd'\right]^{-1}dW\Sigma Wd'\left[dWd'\right]^{-1}\right)
\]
where $d\equiv\frac{\partial m(\theta)}{\partial\theta}$ is the derivative
of the moment function, given as a $K\times M$ matrix in this case.
The derivative is approximated via finite differences, and $\Sigma$
is obtained by obtaining 400 draws from the moment function.\footnote{See function \texttt{get\_stdErrors} in the same julia package \url{https://github.com/floswald/MomentOpt.jl}.}

\subsubsection*{Estimation Sample}

My estimation sample is formed mainly out of averages over SIPP data
moments covering the period 1997\textendash 2012, conditional on non\textendash college
as described above. All moments are constructed using SIPP cross-sectional
survey weights, and all dollar values have been inflated to base year
2012 using the BLS CPI for all urban consumers.\footnote{\url{http://research.stlouisfed.org/fred2/series/CPIAUCSL}}
Averaging over years was necessary to preserve a reasonable sample
size in all conditioning cells. However, it also introduces an initial
conditions and cohort effects problem, since, for example, a 30-year-old
in 1997 faced a different economic environment over their lifecycle
than a similar 30-year-old in 2012 would have. The challenge is to
construct an artificial dataset from simulated data, which has the
same time and age structure as the sample taken from the data \textendash{}
in particular, agents in the model should have faced the same sequence
of aggregate shocks as their data counterparts from the estimation
sample. This requires to simulate individuals starting in different
calendar years, taking into account the actual observed time series
for regional house prices and incomes.

\subsubsection*{Identification}

Identification is achieved by comparing household behaviour under
different price regimes. The variation comes from using the observed
house price and labor productivity series in estimation, which vary
over time and by region. The identifying assumption is that, conditional
on all other model features, households must be statistically identical
across those differing price regimes. In particular, this requires
that household preferences be stable over time and do not vary by
region.

The structural parameters in $\theta$ are related to the moment vector
$m(\theta)$ in a highly non-linear fashion. In general, all moments
in $m(\theta)$ respond to a change in $\theta$. However it is possible
to use graphical analysis to show how some moments relate more strongly
to certain parameters than others.

Regarding parameters of the moving cost function, parameters $\alpha_{0,\tau=0},\alpha_{3},\alpha_{4}$
represent the intercept for low moving cost types, the coefficent
on ownership and the effect of household size on moving costs, respectivley.
They are related to, in order, the average moving rate $E[\text{move}]$,
the moving rate conditional on owning $E[\text{move}|h_{t}=1]$, and
the moving rate conditional on household size $E[\text{move}|s_{t}=1]$.
The age effects $\alpha_{1},\alpha_{2}$ are related to the age\textendash coefficients
of the auxiliary model for moving, defined in expression \eqref{eq:auxmod-move},
as well as the the average proportion of movers in the last period
of life $E[\text{move}|T]$. The relationship between mobility and
ownership, as well as mobility and household size are also captured
by the covariances $Cov(\text{move},h)$ and $Cov(\text{move},s)$,
both of which are again related to the moving cost parameters $\alpha_{3}$
and $\alpha_{4}$.

The proportion of high moving cost types $\pi_{\tau}$ is related
to the data moments concering the number of moves per person, and
in particular the fraction of individuals who never moved, $E[\text{moved never}]$.
The other two moments on the frequency of moves, $E[\text{moved once}]$
and $E[\text{moved twice+}]$ are not part of the moment function,
hence provide out of sample tests.

Given that the house price processes in each region are exogenous
to the model, the parameters measuring utility from ownership, $\xi_{1},\xi_{2}$
are related to a relatively large number of moments: ownership rates
by region and by household size, the covariance of owning with household
size, and the age\textendash profile parameters from the auxiliary
model of ownership in \eqref{eq:auxmod-h}. A crucial parameter in
the model is $\eta$, which measures the scale of consumption in utility:
It informs us how changes in consumption and therefore changes in
income induced by migration, affect payoffs. $\eta$ is nonparametrically
identified from differences in regional mean wages and moving probabilities,
as demonstrated in \cite{kennan2003effect} section 5.4.2.\footnote{Thanks to a referee and the editor for pointing this out to me.}

\subsection{Parameter Estimates and Moments \label{subsec:Parameter-Estimates}}

The model fits the data moments fairly well overall. Figures \ref{fig:fit1}
and \ref{R-fig:fit-2} in the online appendix provide a quick overview
of how the model moments line up with their data counterparts.

The moment vector $m$ contains conditional means and covariances,
which are largely self-explanatory. I introduce two auxiliary models
inluded in $m$ which relate to the age profiles of both migration
and ownership. Both are linear probability models, where the dependent
variable is either ownership status at the beginning of the period,
$h_{it-1}$, or whether a move took place, denoted by $\text{move}_{it}=\mathbf{1}\left[d_{it}\neq d'_{it}\right]$:
\begin{eqnarray}
h_{it} & = & \beta_{0,h}+\beta_{1,h}t_{it}+\beta_{2,h}t_{it}^{2}+u_{h,it}\label{eq:auxmod-h}\\
\text{move}_{it} & = & \beta_{0,m}+\beta_{1,m}t_{it}+\beta_{2,m}t_{it}^{2}+u_{m,it}\label{eq:auxmod-move}
\end{eqnarray}
Several tables constrasting model and data values as well as a detailed
discussion of the fit this provides has been relegated to online
appendix \ref{R-sec:Structural-Model-Fit}.

The estimated parameter vector and standard errors are shown in table
\ref{tab:Parameter-estimates}. It is not possible to attach a simple
interpretation to parameter values in this nonlinear model, however,
it is interesting to identification by looking at the standard errors.
For most parameters, the gradient of the moment function is non-negligible,
and hence, we get precisely estimated coefficients at conventional
levels of statistical significance. The age coefficient in the cost
function, $\alpha_{1}$, is the main non statistically significant
exception to this.

\begin{figure}
\begin{adjustbox}{center}

\includegraphics[scale=0.7]{\Dropbox/model/fit/fit_mobility}\includegraphics[scale=0.7]{\Dropbox/model/fit/fit_ownership}

\end{adjustbox}

\caption{Graphical device to show model fit. These plots show how moments from
data ($x$ axis) line up with moments from simulated data ($y$ axis).
Ideally, all points would lie on the 45 degree line. \label{fig:fit1}
A detailed listing and additional plots are available in online appendix
\ref{R-sec:Structural-Model-Fit}.}

\end{figure}

\begin{table}[p]
\begin{centering}
\input{\Dropbox/model/fit/params.tex}
\par\end{centering}
\caption{Parameter estimates and standard errors.\label{tab:Parameter-estimates}}
\end{table}


\section{Results\label{sec:Results}}

I will now move on to describe the results of this paper. In order
to fully appreciate the results, it is useful to first illustrate
a set of migration elasticities, before answering the question of
why owners move less through the lens of the model. Then I will present
the main set of results pertaining to the value of the migration option.

\subsection{Elasticities with respect to Regional Shocks}

The model can be used to compute elasticities of population size and
migratory flows with respect to regional income shocks. Those elasticities
are an important precursor to the main result of the paper, because
they illustrate the incentives of agents in the event of such a shock.
To measure the elasticity of population or migratory inflows, I simulate
the economy and apply an unexpected and permanent shock to $q_{d}$
in division $d$ in the year 2000. The elasticities are computed by
comparing population size or migration flows across shocked and baseline
scenarios, normalizing the result by the size of the shock.\footnote{Notice that given the cohort setup of the simulator, in this and all
other experiments that involve some notion of a ``shock'', it is
necessary to simluate the model as many times as there are cohorts.
This is so because each cohort experiences the shock at a different
age, and the $p_{dt}$ and $q_{dt}$ are predetermined in the data.
Members of the cohort born in 1985 reach the shock year $t^{*}$ at
a different age than those from the 1984 cohort. The shock is implemented
by immediately changing the policy functions when the shock arises,
and expectations adapt to the new setting. Hence for cohort 1985,
the policy functions look different than for the 1984 cohort, and
so on.}

The results by region are shown in table \ref{tab:elasticities}.
First we observe that the average of population elasticities across
regions is a value around 0.1, implying that on average, a 1\% permanent
increase to regional income will lead to a 0.1\% increase of population
size of the shocked area. Total inflows into regions increase in the
range of approximately 0.8\% to 1.9\%, the inflow rate of renters
increases more than the one of incoming buyers throughout. The next
set of columns looks at the complement to those statistics, i.e. the
elasticity of outflows. In the present case of a positive income shock,
outflows decline in general as both renters and owners are less likely
to move away. Table \ref{R-tab:elasticities-p} in the online appendix
shows the corresponding elasticities for the case of a positive regional
house price shock.

\begin{table}[H]
\begin{centering}
\input{\Dropbox/model/fit/elasticities_q.tex}
\par\end{centering}
\caption{Elasticities with respect to an unexpected and permanently positive
income shock by region. This table reports elasticities of population
(i.e. the stock of individuals present in each period) and migration
\emph{inflows} and \emph{outflows }elasticities. For example, the
percentage change in renter inflows is defined as $\frac{\#[\text{move to }d\text{ as renter}|\text{shock}]-\#[\text{move to }d\text{ as renter}|\text{no shock}]}{\#[\text{move to }d\text{ as renter}|\text{no shock}]}$
in each period, similarly for owners and for outflows. Elasticities
are computed as averages over all years after the shock occurs. \label{tab:elasticities}}
\end{table}


\subsection{Why Do Owners Move Less?}

There are several reasons for why owners move less than renters. First,
they have higher moving costs as implied by a positive estimate for
parameter $\alpha_{3}$. Second, owners pay a transaction cost each
time they sell the house (proportional cost $\phi$), so any (expected)
gains from migration need to be traded off against this financial
cost. Third, owners have to comply with the downpayment constraint
if they wish to buy in the new region, which puts restrictions on
the consumption paths of movers. Most owners will indeed return to
ownership status in the new region, given the utility benefits, and
given that in principle they are above the downpayment constraint.
Fourth, ownership is correlated with larger household size $(s=1)$,
which itself carries a higher moving cost $(\alpha_{4})$. Last but
certainly not least, a large proportion of owners is of the stayer
type because they self-select into ownership as was discussed in section
\ref{subsec:Moving}. The ownership rate conditional only on moving
cost type is 0.59 for movers $(\tau=0)$ vs 0.64 for stayers $(\tau=1)$.

To investigate those issues in more detail, I now sequentially remove
owner-specific moving costs in table \ref{tab:Decomposing-owner's-moving}
from the model. We should imagine an approximation to the partial
derivative of the moment function of the model with respect to the
parameters $\alpha_{3}$ and $\phi$. If we see a large reaction of
the model\textendash generated moments after setting a certain cost
component to zero, we can conclude that this component is relatively
important to explain the data. Starting therefore in the top panel
of table \ref{tab:Decomposing-owner's-moving}, we see changes in
three key moments when considering first all types of households.
The first row shows the percentage change in the aggregate ownership
rate for different configurations of the model parameters. Setting
the utility cost of moving for owners to zero in the column labelled
$\alpha_{3}=0$ increases the ownership rate by 5.6\%, because this
makes owning a more attractive option in case of the need to migrate.
Similarly, abolishing the financial transaction cost from selling
in column $\phi=0$ leads to an increase of 2.7\%. In the final column
$\alpha_{3}=\phi=0$, combining both changes, we see an increase of
8.2\% in ownership. 

The second row shows the same experiment for the overall migration
rate. Here the direct impact of $\alpha_{3}=0$ is larger than removal
of the financial transaction cost, and it leads to roughly a 4\% increase
in overall migration. The third row finally conditions on owners in
order to emphasize that most migratory movement comes from renters
to start with, so the previous experiment masks a great deal of heterogeneity.
Here we see a large increase of 140\% in the migratory propensity
of owners after removing the $\alpha_{3}$ cost component. Overall,
the first panel shows that both cost components affect both ownership
and migration simulateneously, and that the impact of removing $\alpha_{3}$
is more important.

The bottom panel of the table repeats this exercise while conditioning
only on the \emph{mover }population, i.e. types $\tau=0$. This helps
to further clarify the top panel, where the results are from a mixture
of stayers and movers. It is interesting to note that among the movers,
the impacts on ownership rate are greater than for the population
at large, whereas those on migration of owners are smaller throughout.
The former is a consequence of \emph{stayers }being overrepresented
in the group of owners already, hence we see a smaller increase in
ownership in the top panel. The latter effect has to do with this
change in composition of renters versus owners: given that in the
mover population we register a larger increase in ownership, the denominator
in \emph{Migration | Own }becomes larger, hence the value decreases.
Finally, the fact that the increase in migration rate is identical
across both panels shows that any additional mobility can come only
from the subpopulation of mover types.

The conclusion from this exercise is that including utility cost $\alpha_{3}$
over and above the assumed financial transaction cost $\phi$ (6\%)
in the model is important in order to fit the data, both in terms
of propensity to move and in terms of changes in the ownernship rate.

\begin{table}
\begin{centering}
\input{\Dropbox/model/experiments/decompose_MC/decompose_MC.tex}
\par\end{centering}
\caption{Decomposing owners' moving costs. This table shows the percentage
increase in three key model moments related to ownership and migration
as we successively remove moving costs for owners. Compares baseline
statistics to scenarios with no additional moving cost for owners
($\alpha_{3}=0$), no financial transaction costs from selling the
house $(\phi=0)$, and neither of the two $(\alpha_{3}=\phi=0)$.
The top panel is for the entire population, the bottom panel conditions
on mover types $(\tau=0)$ only.\label{tab:Decomposing-owner's-moving}}
\end{table}


\subsection{Owner Regret}

Individuals in the model decide whether to buy a house or rent based
on a multi\textendash facetted tradeoff involving the size of available
houses, their preference for owned property and their expectations
about future moves and prices. Buying a house is consequential for
later mobility decisions, as we have seen, because owners face higher
moving costs. A pertinent question question in this context is whether
owners actually \emph{regret }having bought their house, if the state
of the world changes in an unforeseen way? For example, imagine that
expectations about future prices were wrong, in the sense that there
is an unexpected shock. Imagine further that that owner finds themselves
with a greatly devalued house \textendash{} how big is the cost of
being an owner in this new circumstance? 

We want to assess this cost by looking at how much an owner in the
shocked scenario would be willing to pay to become a comparable renter.
Establishing comparability is important, because in general owners
are of higher networth than renters (by virtue of the downpayment
requirement and subsequent capital gains). Therefore the experiment
proceeds by establishing an asset level $a^{*}$ in the baseline such
that, with some abuse of notation, $V^{own}(a^{*})=V^{rent}(0)$,
in other words where an owner's value is equal to a renter's value
with zero assets. $a^{*}$ is in general a negative number, proportional
to the greater networth of owners and the additional utility derived
from owning. Then, we shock either $q$ or $p$ at a certain age in
a given region, and we want to know how much the owner would be willing
to pay in order to convert to a renter, measured at the pre-shock
reference level $a^{*}$.\footnote{Note that this experiment measures something different from the value
of the option \emph{sell }of the owner. Since, if it was optimal to
become a renter by selling, they would of course make this choice.
Here we know that before the shock, the owner is as well-off at $a^{*}$
as the zero asset renter, and we want to know how this relationship
changes in the shocked economy, again measured at $a^{*}$.} 

The results of this exercise are displayed in table \ref{tab:Owner-WTP}
for a permanent and unexpected reduction in the level of either $q$
or $p$ of 10\% in a given Division in year 2000. We observe first
that in the column corresponding to the $q$ shock, there are relatively
small dollar amounts (table is in 1000 of dollars), and some are negative.
The negative entries imply that the owner would not be willing to
pay anything after income drops by 10\%. The intuition here is that
in the case of an income shock, the renter is equally affected, hence
there is only a small desire to become, or no desire at all to become,
a renter with zero assets and a significantly reduced income. This
is also the case because owners still have their housing capital to
fall back on, which is unaffected from the shock in this scenario.
While we saw previsously in table \ref{tab:elasticities} that renters
respond stronger to shocks than owners in terms of increased emigration,
for obvious reasons, the expected gains in terms of lifetime value
do not seem to compensate an owner to want to switch to that particular
type of renter in most regions and take advantage of less costly migration.

Quite different to this is the outcome of a price shock. We see in
the first column of table \ref{tab:Owner-WTP} that owners would be
willing to pay amounts ranging from about 14,700 and up to 25,800
dollars in order to be converted to a renter with zero assets after
the unexpected price reduction. This experiment shares some features
of mortgage default (which is not modeled), in that the owner evaluates
a ``reset'' option here: the mortgage debt burden is increased substantially
by the price drop, so much so that the owner would be willing to pay
substantial amounts to get converted to a renter with zero assets.
Thus, to give an answer to the initial question of how much owners
regret to having bought when things turn out not as expected, or on
the contrary, how much they would value being a ``free but asset-poor''
renter again in case of a price shock, one could say that this lies
in between 15 and 25 thousand dollars. 

\begin{table}
\begin{centering}
\input{\Dropbox/model/experiments/ownersWTP2.tex}
\par\end{centering}
\caption{Owners willingness to pay to convert to a comparable renter after
an unexpected 10\% reduction in price or income arises. In 1000 of
dollars.\label{tab:Owner-WTP}}

\end{table}


\subsection{The value of Migration\label{subsec:The-value-of}}

This section presents a measure of how much individuals care about
having the option to migrate across regions. The question is motivated
by relatively low migratory flows across regions, 1.32\% of households
per year, as initially stated. Do low flows imply low value? And how
does this valuation depend on age, location, ownership status, and
current state of the business cycle? This value is related to what
in \cite{yagan} is called \emph{migratory insurance.} Here, I do
not infer this from the amount of migration after a local shock has
occured, ex post, but I consider how much individuals value to have
the option to migrate ex ante, in case a shock were to occur.

I will attempt to answer this question by first simulating the model
in the baseline equilibrium, i.e. at observed regional prices and
incomes and importantly, \emph{with migration} as an option. Subsequently
I will compare this to a counterfactual equilibrium \emph{without
}the migration option, i.e. migration is shut down in the entire economy.\footnote{Even though trade is absent from the model, it helps the interpretation
to assume that existing trade channels between regions remain intact
throughout the experiment, i.e. we only expect changes from the absence
of individuals changing location.} The welfare measure in terms of compensating consumption is defined
in appendix \ref{R-sec:Welfare-Measure}.

Before delving into the results it is paramount to clarify the role
of the partial equilibrium assumption in this counterfactual.\footnote{Thanks to an anonymous referee for helping me to clarify this point.}
Assuming that the model is well-specified, the structural parameters
are such that given prices and incomes, the resulting decision rules
of agents in the model are correct. This mapping from model to data
was shown to be satisfactory in section \ref{subsec:Parameter-Estimates}.
What this model cannot deliver is a prediction of how the \emph{exogenous}
series for $q$ and $p$ would change if we were to abolish migration.
To address this concern at least to some degree, I will present different
scenarios of this counterfactual, a baseline version with prices unchanged,
and a set of experiments with changed prices in the appendix. There
is no direct empirical guidance on the effects of such a drastic experiment
on regional house prices and labor income levels, except maybe that
\emph{in general, }wage effects of immigration are \emph{small}.\footnote{I take this insight from the literature that assesses the impact of
immigrants on native wages, exemplified by, for example, \cite{dustmann2008labour,dustmann2013effect,card2012comment},
which finds negligible negative impacts of immigrants on wages of
low-skilled workers, and slightly positive ones elsewhere along the
income distribution. Needless to say that the current model is a much
simplified version of those studies in terms of skill composition
of the labor force and indeed wage determination \textendash{} it
abstracts from immigrants (i.e. different skills groups and wage determination)
altogether, hence it is probably too remote in order to directly use
their results.} Absent such empirial evidence, I try to cover the most relevant cases.
Version two thus decreases both regional incomes and prices by 1\%
relative to their observed trajectories after the shutdown of migration.
This could arise as a result of decreased firm productivity from the
lack of suitable (migrant) skills, which leads to lower disposable
income in a given region and hence a reduction in house prices. Version
three simulates a bust scenario, where incomes decline by 5\% and
house prices by 10\%. Summing up, this experiment is available in
three versions:
\begin{enumerate}
\item Baseline $\{q_{dt},p_{dt}\}_{t=1997}^{2012}$: Loss of migrants has
negligible impact on regional prices.
\item -1\% shock to $\{q_{dt},p_{dt}\}_{t=1997}^{2012}$: Local productivity
suffers a small loss.
\item -5\%/-10\% shock: Large productivity decline and amplified effect
on house prices.
\end{enumerate}
Starting with the scenario where prices are unaffected by the experiment,
table \ref{tab:shutdown-cons-base} provides the main results of the
paper. In the first row the consumption compensation demanded in the
entire economy, for different subsets of the population, including
young, old, owners at age 30, renters at age 30, individuals whose
average $z$ history is lower than the 20-th percentile of the distribution
of $z$ histories (i.e. \emph{poor }individuals), same for people
above thee 80-th percentile, and finally the entire population. The
values in the table stand for the per period increase in consumption
that would make individuals indifferent between baseline and migration
shutdown, as a percentage of what they had optimally chosen to consume
in the policy environment. Hence it is a measure for their willingness
to pay to maintain migration. For example, in the first row of table
\ref{tab:shutdown-cons-base}, the group of young people (with age
below half of their lifespan) would demand an increase of 30.8\% of
optimal per period consumption. This amount is lower for old people
at 8\%, which is intuitive since they forgo fewer periods where migration
could have been optimally chosen. We can look at the same measure
by ownership status at age 30: owners at that age demand 4.2\% more
consumption, while same aged renters demand more, i.e. 21.7\%. Part
of this difference comes from the fact that stayer types know that
they will not move, hence are more likely to buy, hence suffer less
from a removal of the migration option. The next two columns condition
the measure on position in the distribution of realized $z$ draws.
It is evident that people who have relatively favourable draws of
$z$, value migration more in most regions, which has to do with the
shape of the estimated transition matrix for movers, $G_{\text{move}}$
as illustrated in appendix \ref{R-sec:Copula-Estimation}. (High $z$
movers can expect another high $z$ draw in the new location.) As
an average over all individuals treated in this experiment (column
labelled ATE), the corresponding number is 19.2\% of consumption compensation
demanded. 

In the same table we continue with this exercise and split the sample
by region to understand how heterogeneous those valuations are distributed.
Perhaps not surprisingly, there is a lot of variation in how individuals
feel about removal of the migration option by region. In general,
living in a high-income, high-price region like Pacific accentuates
the difference between young and old even more (54.7\% vs -7.8\% compensation
demanded). Put simply, this is because high prices are good for owners,
who are more likely to stay in the region no matter what, but bad
for renters who cannot afford to buy a house. They would prefer to
migrate at some point if necessary and therefore suffer disproportionately
from the removal of the migration option. By way of summary of this
table, individuals value having the option of migration to a large
degree and in the range of -11\% (owners aged 30 in South Atlantic)
and up to 60\% of per period consumption. Negative entries imply an
\emph{un}willingness to pay for the migration option: this applies
to groups who are (young) owners in high-price regions.

Versions two and three of this experiment are described in appendix
\ref{R-subsec:Migration-Shutdown}, where the general conclusions
from this series of results goes through, with the qualification that
the bigger the price shock after migration shutdown, the more individuals
value the baseline. This sequence of counterfactuals tells us that
the estimates from the baseline scenario with observed prices provide
a lower bound. If in reality a general equilibrium effect would change
regional incomes and prices downwards, the valuation of the migration
option would be larger than displayed in table \ref{tab:shutdown-cons-base}.

\begin{table}
\begin{centering}
\input{\Dropbox/model/experiments/noMove_region_z_ctax.tex}
\par\end{centering}
\caption{Consumption compensation demanded after migration shutdown in the
baseline scenario (regional prices are unaffected by the shutdown
of migration). The numbers in this table represent the required percentage
scaling factor $\Delta c$ by which optimal consumption under the
shutdown policy would have to be increased in order for individuals
to be indifferent to the baseline. Positive values indicate people
disapproving the policy, negative values indicate the opposite. The
columns are, in order, \emph{Young }(population below half of total
lifetime $J$), \emph{Old }(complement to \emph{Young})\emph{, own,30
}(population who owns at age 30), \emph{rent,30 }(population who rents
at age 30), $z_{q}$\emph{ }(population below/above the $q$-quantile
of the lifetime idiosyncratic income shock distribution). \emph{A}
value of 30.8\% as in the first row of the second column means that
young people would demand an increase of 30.8\% of the consumption
level which they had optimally chosen under the policy, in order for
them to be indifferent.\label{tab:shutdown-cons-base}}
\end{table}


\subsection{Effects along the Lifecycle and by Ownership Status}

We have shown that lifetime utility changes dramatically with the
removal of migration. We now go further and investigate where those
changes come from, i.e. how do the state variables of individuals
in the model change? In the following, we focus only on the main counterfactual
with constant prices.

Starting out with the lifecycle considerations, the results are presented
in figure \ref{fig:noMove-lifecycle}. The first panel in the top
row in some ways repeats the insights from the previous section: younger
individuals suffer particularly, experiencing a loss as in lifetime
utility of almost 4\% in the first period of life. As time goes by,
the losses get smaller until they vanish towards the end of the lifecycle.
We observe in the next panel that removing migration implies a substantial
drop in average income at all ages. This implies that some profitable
moves in terms of better wage draw could not be completed as a result
of the policy. 

The next two panels for $a$ and $h$ are best viewed in conjunction,
as they are tightly connected: After migration is abolished, the aggregate
homeownership rate increases strongly for 30 year-olds, and with it
the outstanding mortgage balance as measured by negative net assets
$a$. In light of the fixed regional price series $\{q_{dt},p_{dt}\}_{t=1997}^{2012}$
this result may be somewhat surprising. We observe that after migration
is abolished, homeownership rises by about 39\% at age 30. Notice
that this increase implies that far more than only the previous migrants
now choose to buy, as this was relatively small group of people. The
increase in ownership comes from the fact that potential future moves\emph{
}of \emph{all individuals} have been ruled out, and therefore a much
larger number of them finds it profitable to take out a mortgage and
buy in the current (constant) location. This suggests that absent
the option to move to a better region in response to shocks, the best
thing to do is is to invest in more enjoyable housing in the current
region. In particular, this implies that the aggregate mortgage balance
increases dramatically, hence the $\%\Delta a$ panel shows negative
percentage changes for most ages. The non-monotonicity in that figure
occurs because around age 37 the group of mortgage holders is considerably
poorer than in the baseline, where the asset balance starts to turn
positive for individuals with high labor incomes of that age.

\begin{figure}
\begin{centering}
\input{\Dropbox/model/data_repo/out_graphs_jl/noMove_ps_1.0_ys_1.0_age.tex}
\par\end{centering}
\caption{Impact of \emph{migration shutdown }along the lifecycle. Each line
shows the percentage difference between baseline conditional mean
of a certain variable of interest and the conditional mean under the
policy. The means are conditional on age. The labels stand for, in
order:$v$ value function, $y$ is individual labor income, $a$ is
asset position and $h$ is the ownership status. \label{fig:noMove-lifecycle}}
\end{figure}

In figure \ref{fig:noMove-age30} I look at the experiment by measuring
the effects for individuals who did and did not own their house by
the age of 30. It shows that the bulk of the previous lifecyle results
must be driven by young renters, as opposed to owners. Young owners
in figure \ref{fig:noMove-age30} have a smaller utility loss, as
shown in panel $v$ which displays lifetime utility. The second panel
again illustrates that it is young renters who miss out on profitable
moves in terms of individual income $y$. The remaining panels for
$a,h$ and total wealth $w$ all show once more that young renters
take on more mortgage debt in order to buy houses in their current
(now, permanent) region.

In summary, the results in this section show that moving costs differ
greatly by ownership status; that the resulting elasticities of migration
with respect to regional shocks differ greatly as well; and finally,
that individuals place a large consumption value on having the option
to migrate across Census Division borders, ranging from negative values
(some groups prefer not to have the option) and up to about 19\% of
per period optimal consumption.

\begin{figure}
\begin{centering}
\input{\Dropbox/model/data_repo/out_graphs_jl/noMove_ps_1.0_ys_1.0_own30.tex}
\par\end{centering}
\caption{Impact of \emph{migration shutdown }by ownership status at age 30.
Bars show the percentage difference between the conditional mean of
the respective variable conditional on owernship status at age 30
in the baseline, and the shutdown policy.\label{fig:noMove-age30}The
labels stand for, in order: $v$ value function, $u$ period payoff,
$y$ is individual labor income, $a$ is asset position, $h$ is the
ownership status, and $w$ is total wealth.}
\end{figure}

\newpage

\section{Conclusion}

The main result of this paper is to show that despite average migration
rates being low, the option value associated with the possibility
to leave a location in a world with regional shocks to house prices
and labor income is large. Removing the option to migrate in the model
implies an associated reduction in per period consumption that ranges
from 0\% (or even implied increases in consumption) up to 19\%
depending on the type of household one considers. Variations in this
measure vary widely with age, ownership status, original location,
and point in the business cycle. To arrive at this result, I construct
a lifecycle model which includes homeownership as a choice variable
next to savings and location choices, which I then fit to SIPP data
and use to make counterfactual experiments. Considering homeownership
is motivated by the fact that well over 60\% of the US population
are owners, and the observation that owners exhibit vastly different
migratory behaviour than renters. The model places particular emphasis
on a close representation of the observed house price and income series,
both of which exhibit strong correlation of regional shocks. These
results resonate with the findings in \cite{yagan} in the sense that
both papers provide an estimate of \emph{migratory insurance}. Here
I provide a well-defined value in terms of consumption, whereas the
unit of measurement is more abstract in \cite{yagan}.

The present model delivers structural estimates for differential moving
costs between owners and renters, over and above financial transaction
costs. It is shown that this moving cost component is first order
in explaining differential moving rates between renters and owners.
The results are important for policy makers in terms of both housing
and labor markets, since they illustrate the implied costs of policies
which might reduce mobility, such as implicit or explicit subsidies
to homeownership, for example.

\newpage

\bibliographystyle{ecta}
\bibliography{references}

\end{document}
