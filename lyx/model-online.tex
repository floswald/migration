%% LyX 2.3.2 created this file.  For more info, see http://www.lyx.org/.
%% Do not edit unless you really know what you are doing.
\documentclass[11pt,english]{article}
\usepackage[T1]{fontenc}
\usepackage[latin9]{inputenc}
\usepackage[a4paper]{geometry}
\geometry{verbose,lmargin=2cm,rmargin=2cm}
\setlength{\parskip}{\smallskipamount}
\setlength{\parindent}{0pt}
\usepackage{color}
\usepackage{babel}
\usepackage{float}
\usepackage{rotfloat}
\usepackage{url}
\usepackage{amsmath}
\usepackage{amssymb}
\usepackage{graphicx}
\usepackage{setspace}
\onehalfspacing
\usepackage[unicode=true,pdfusetitle,
 bookmarks=true,bookmarksnumbered=false,bookmarksopen=false,
 breaklinks=true,pdfborder={0 0 0},pdfborderstyle={},backref=false,colorlinks=true]
 {hyperref}

\makeatletter
%%%%%%%%%%%%%%%%%%%%%%%%%%%%%% Textclass specific LaTeX commands.
\numberwithin{figure}{section}
\numberwithin{table}{section}
\numberwithin{equation}{section}

%%%%%%%%%%%%%%%%%%%%%%%%%%%%%% User specified LaTeX commands.
\usepackage{natbib}
\usepackage{url}
\usepackage{booktabs}
\usepackage{dcolumn}
\usepackage{longtable}
\usepackage{xcolor}
\definecolor{dark-red}{rgb}{0.8,0.15,0.15}
\definecolor{dark-blue}{rgb}{0.15,0.15,0.8}
\definecolor{medium-blue}{rgb}{0,0,0.5}
\hypersetup{
    colorlinks, linkcolor={dark-red},
    citecolor={dark-blue}, urlcolor={medium-blue}
}
\newcommand{\Dropbox}{${HOME}/Dropbox/research/mobility/output}
\usepackage{tikz}
\usepackage{adjustbox}
\usepackage{pgfplots}
\usepackage{makecell}
\usepackage{xr}
\externaldocument[M-]{model}

%% definitions
\def\thesection{\Alph{section}}

\makeatother

\begin{document}
\title{Online Appendix to \\
The Effect of Homeownership on the Option Value of Regional Migration}
\author{Florian Oswald\thanks{email: \protect\url{florian.oswald@gmail.com}. }\\
Sciences Po}
\date{\today}
\maketitle
\begin{abstract}
This is the online appendix with additional material to the version
appearing in Quantitative Economics.
\end{abstract}
\textbf{JEL-Codes: }J6, R23, R21

\section{Introduction and Notation}

This is the online appendix for \emph{The Effect of Homeownership on
the Option Value of Regional Migration }available on my website at \url{https://floswald.github.io/pdf/homeownership-appendix.pdf} as well as on the dedicated website of Quantitative Economics \url{http://qeconomics.org/}.
In this document I number sections alphabetically (A, B, ...) and
equations with roman numbers (I, II, ...). Standard latin numbering
(1, 2, ...) refers to the main text.

\section{Dimensionality Reduction: Factor Model\label{sec:DimReduce}}

This section is concerned with the dimensionality reduction of regional
house prices and income time series, as undertaken in the main text
in section \ref{M-subsec:Dim-Reduce}.

\subsection{Data Description and Problem Outline\label{subsec:DimReduce-Data}}

Here we present detailed results for the regional house price and
income data. This part is related to section \ref{M-subsec:Income-not-IID} in the main text. The main issue we face can easily be illustrated with a series of figures which show the time series of regional prices and incomes. Starting with figures \ref{fig:Regional-House-Price-data} we see the relationship between a national house price index with it's regional counterparts. Figure \ref{fig:Regional-Income-data} shows the same for the regional income data. What is noteworthy in both cases is that each regional series seems strongly correlated with the national series, however, each region in a different kind of way.


\begin{figure}
\begin{centering}
\includegraphics[scale=0.8]{/Users/florian.oswald/Dropbox/research/mobility/output/data/sipp/agg_reg_p}
\par\end{centering}
\caption{Regional House Price indices vs National average. FHFA house price
index.\label{fig:Regional-House-Price-data}}
\end{figure}

\begin{figure}
\begin{centering}
\includegraphics[scale=0.8]{/Users/florian.oswald/Dropbox/research/mobility/output/data/sipp/agg_reg_y}
\par\end{centering}
\caption{Regional per capita personal income $q_{dt}$ from BEA vs the national
average index $Q$, for which I use real per capital GDP.\label{fig:Regional-Income-data}}
\end{figure}

% \begin{sidewaysfigure}
% \includegraphics[scale=0.6]{/Users/florian.oswald/Dropbox/research/mobility/output/data/FHFA/detrended-p}

% \caption{Detrended $p$ time series.\label{fig:Detrended-p}}
% \end{sidewaysfigure}

% \begin{sidewaysfigure}
% \includegraphics[scale=0.6]{/Users/florian.oswald/Dropbox/research/mobility/output/data/FHFA/detrended-y}

% \caption{Detrended $q$ time series.\label{fig:Detrended-y}}
% \end{sidewaysfigure}

Tables \ref{tab:Cross-corr-q} and \ref{tab:Cross-corr-p} give the cross-correlation of the detrended series from the preceding plots. We observe that those are large throughout. Finally, table \ref{tab:Autocorrs} shows that each series independently is very persistent by measuring their first order partial autocorrelation coefficients.

\begin{table}
\begin{centering}
\input{\Dropbox/data/FHFA/y_corrs.tex}
\par\end{centering}
\caption{Cross-correlations between detrended $q$ series\label{tab:Cross-corr-q}}
\end{table}

\begin{table}
\begin{centering}
\input{\Dropbox/data/FHFA/p_corrs.tex}
\par\end{centering}
\caption{Cross-correlations between detrended $p$ series\label{tab:Cross-corr-p}}
\end{table}

\begin{table}
\begin{centering}
\input{\Dropbox/data/FHFA/auto_corrs.tex}
\par\end{centering}
\caption{First order partial autocorrelation coefficients of both $q$ and
$p$ from raw (i.e. not detrended) time series.\label{tab:Autocorrs}}
\end{table}




\subsection{Factor Model}

Remember that the factor model relating aggregate factors $\mathbf{F}$
to regional prices is 
\begin{eqnarray}
\mathbf{F}_{t} & = & A\mathbf{F}_{t-1}+\nu_{t-1}\nonumber \\
\nu_{t} & \sim & N\left(\begin{bmatrix}0\\
0
\end{bmatrix},\Sigma\right)\label{eq:VAR-process-agg}\\
\mathbf{F}_{t} & = & \begin{bmatrix}Q_{t}\\
P_{t}
\end{bmatrix}\nonumber 
\end{eqnarray}
and that the mapping into regions $d$ is deterministally given assumed
to be
\begin{eqnarray}
\begin{bmatrix}q_{dt}\\
p_{dt}
\end{bmatrix} & = & \mathbf{a}_{d}\mathbf{F}_{t}.\label{eq:VAR-reg-determ}
\end{eqnarray}
The empirical implementation estimates $\mathbf{a}_{d}$ in a SUR
model:
\begin{eqnarray}
\begin{bmatrix}q_{dt}\\
p_{dt}
\end{bmatrix} & = & \mathbf{a}_{d}\mathbf{F}_{t}+\eta_{dt}\nonumber \\
\eta_{dt} & \sim & N\left(\begin{bmatrix}0\\
0
\end{bmatrix},\Omega_{d}\right)\label{eq:VAR-process-1}
\end{eqnarray}


\subsection{Model Estimates and Performance}

\begin{figure}
\begin{centering}
\includegraphics[scale=0.8]{../../../Dropbox/research/mobility/output/data/sipp/VAR_reg_p}
\par\end{centering}
\caption{This figure shows the observed and predicted time series for mean
income by Census Division. The prediction is obtained from the VAR
model in \eqref{eq:VAR-reg-determ}, which relates the aggreate series
$\left\{ Q_{t},P_{t}\right\} _{t=1968}^{2012}$ to mean labor productivity
$\left\{ q_{dt}\right\} _{t=1968}^{2012}$ for each region $d$\emph{.
}Agents use this prediction in the model, i.e. from observing an aggregate
value $\mathbf{F}_{t}=\left(P_{t},Q_{t}\right)$ they infer a value
for $q_{dt}$\emph{ }for each region above. \emph{\label{fig:mapping-P-p}}}
\end{figure}

\begin{table}
\begin{centering}
\input{\Dropbox/data/sipp/VAR1-edit.tex}
\vspace{1cm}
\par\end{centering}
\begin{centering}
\input{\Dropbox/data/sipp/VAR2-edit.tex}
\par\end{centering}
\caption{Aggregate to Regional price mappings. This table shows the estimated
coefficients from equation \eqref{eq:VAR-reg-determ}, which relates
the aggregate factors $\left(Q_{t},P_{t}\right)$ to regional income
and house price $\left(q_{dt},p_{dt}\right)$. \label{tab:Aggregate-to-Regional-VAR}}
\end{table}


\subsection{Transformation of Aggregate to Regional Shocks\label{subsec:Transformation-of-Aggregate}}

To investigate how aggregate shocks are translated into regional shocks,
I fix $\mathbf{F}_{t}$ at its mean value except for $t=2000$ when
I shock component $Q_{t}$ by $-10\%$ ($P_{t}=\overline{P}$ throughout
this exercise). The transformation of this into regional deviations
of $q_{dt}$ are displayed in figure \ref{fig:shock-reg-model-illustrate}.
This shows that the model generates considerable variation in the
size of the resulting local shock, which is a desirable feature. A
similar size regional shock in all regions would not seem very realistic.

\begin{figure}
\begin{adjustbox}{center}

\includegraphics[scale=0.8]{/Users/florian.oswald/Dropbox/research/mobility/output/data/sipp/impulse_y}

\end{adjustbox}

\caption{Illustrating the transformation of aggregate shocks into regional
counterparts. This exercise keeps the aggregate $\mathbf{F}_{t}$
constant at its mean level except for period $t=2000$ where the $Y_{t}$
component (only) is reduced by 10\% relative to its mean.\emph{ }The
panels show the resulting deviation in regional labor productivity
$q_{dt}$\emph{. \label{fig:shock-reg-model-illustrate}}}
\end{figure}


\subsection{State level vs Division level\label{subsec:State-Division}}

Tables \ref{tab:state-vs-region-p} and \ref{tab:state-vs-region-y}
show the results from regressions of the form
\begin{align*}
q_{st} & =\beta_{0}+\beta_{1}q_{dt}+u_{st},s\in d,t=1967,\dots,2012\\
p_{st} & =\alpha_{0}+\alpha_{1}p_{dt}+u_{st},s\in d,t=1967,\dots,2012
\end{align*}

The aim of those regressions is to measure how much state level variation
is captured by the corresponding Divison level indices $q_{dt}$ and
$p_{dt}$.

\begin{table}
\begin{centering}
\input{\Dropbox/data/FHFA/rsquareds.tex}
\par\end{centering}
\caption{$R^{2}$ from pooled OLS regression of state level indices $p_{st},q_{st}$
on corresponding Division level indices $p_{dt},q_{dt}$.\label{tab:-region-state-rsquared}}
\end{table}

\begin{sidewaystable}
\input{\Dropbox/data/FHFA/p-P-mods-edit.tex}

\caption{state vs region level price indices.\label{tab:state-vs-region-p}}
\end{sidewaystable}

\begin{sidewaystable}
\input{\Dropbox/data/FHFA/q-Q-mods-edit.tex}

\caption{state vs region level price indices.\label{tab:state-vs-region-y}}
\end{sidewaystable}


\section{Individual Income Process}

\begin{sidewaystable}
\begin{centering}
\input{\Dropbox/model/fit/region_2_indi_y-edit.tex}
\par\end{centering}
\caption{Regional Mean Income to Individual level income mapping. This is the
empirical implementation of equation \eqref{M-eq:indiv-labor-empirical}
in the main text, as explained in section \ref{M-subsec:Individual-Income-Process}.
The estimated equation is $\log y_{idt}=\beta_{0}+\eta_{d}\log\overline{y}_{dt}+\beta_{1}\text{age}_{it}+\beta_{2}\text{age}_{it}^{2}+\beta_{3}\text{age}_{it}^{3}+u_{it}$
and the coefficients $\eta$ are denoted with the Division names.
\label{tab:Regional-2-indiv-income}}
\end{sidewaystable}

\begin{figure}
\begin{centering}
\input{\Dropbox/model/fit/income_profiles.tex}
\par\end{centering}
\caption{Age profiles as predicted by the empirical implementation of individual
labor income, equation \eqref{M-eq:indiv-labor-empirical}, for three
different levels of regional mean productivity $q$. Notice that in
the model as well as in the data it is never the case that all regions
have the same level of average income.\label{fig:age-profiles}}
\end{figure}


\subsection{Estimation of $G_{move}$\label{sec:Copula-Estimation}}

In a first step I estimate the marginal distributions of $z_{idt}$
and $z_{ikt+1}$ for all movers. These are the cross-sectional distributions
of residuals $z_{it}$ and $z_{it+1}$ from the regression in expression
\eqref{eq:copula-margins-1}, which is estimated for all movers. The
move takes place in period $t$, such that by assumption, $z_{it}$
is the residual wage in origin location $d$, and $z_{it+1}$ is the
residual wage in the new location $k$. The proceedure relies crucially
on the assumption that individuals have to move to the new region
before they can discover $z_{t+1}$. One could account for apotential
selection effect on $z_{t}$ by moving estimation of this part into
the structural model and jointly estimate behavioural and wage related
parameters. The model provides a set of exclusion restrictions that
would allow to do this in theory. Identification of a potential selection
effect may be difficult, however, because the sample of movers is
relatively small. 
\begin{equation}
\ln y_{idt}=\beta_{0}+\beta_{1}\text{college}_{it}+\delta p(\text{age}_{it})+\beta_{2}\text{numkids}_{it}+\beta_{3}\text{sex}_{it}+\beta_{4}\text{metro}_{it}+\gamma_{d}+u_{it}\label{eq:copula-margins-1}
\end{equation}
Here $p(\text{age})$ is a third order polynomial in age, metro is
an indicator for metropolitan status and $\gamma_{d}$ is a Division
fixed effect. Remember that the copula is given as 
\[
C(u_{1},u_{2})=F\left(F_{1}^{-1}(u_{1}),F_{2}^{-1}(u_{2})\right)
\]
so that it is necessary to specify 1) the copula family and 2) both
margins $F_{1},F_{2}$. Visual inspection of the margins lead me to
assume normal margins, see figure \ref{fig:Densities-of-wage-z}.
Estimation itself is based on the respective rank of $z$ in the empirical
distributions. Denoting the standardized values by $\left(\hat{u}_{it},\hat{u}_{it+1}\right)$,
the next step involves fitting the a normal copula via maximum likelihood
to this data. The results are shown in table \ref{tab:Normal-Copula-estimates},
and they indicate a correlation between $\hat{u}_{it}$ and $\hat{u}_{it+1}$
of 0.59. This estimate together with the marginal distibutions of
$z_{it}$ and $z_{it+1}$ are used in the structural model, where
I use the current value of $z$, evaluated in the marginal distribution
of $z_{it}$ for a mover together with the copula estimate $\hat{G}_{\text{move}}$
to draw the next value of $z'$. The contours of the corresponding
density function of copula $C$ are shown in figure \ref{fig:Contours-of-copula}.

\begin{table}
\begin{centering}
\input{\Dropbox/model/fit/copulas-edit.tex}
\par\end{centering}
\caption{Normal Copula estimates for the rank of wage residuals $u_{it}$ and
$u_{it+1}$ for individuals who move in period $t$. The algorithm
was not able to compute a standard error for $\rho$ because of a
flat hessian.\label{tab:Normal-Copula-estimates}}
\end{table}

\begin{figure}
\begin{centering}
\includegraphics[scale=0.75]{../../../Dropbox/research/mobility/output/data/sipp/z_margins}
\par\end{centering}
\caption{Densities of wage residual $u$in equation \eqref{eq:copula-margins-1}
of movers today $(u)$ and tomorrow (u\_plus1).\label{fig:Densities-of-wage-z}}
\end{figure}

\begin{figure}
\begin{centering}
\includegraphics[scale=0.8]{../../../Dropbox/research/mobility/output/model/data_repo/out_graphs_jl/mover-copula}
\par\end{centering}
\caption{Contours of copula density which is the estimate of the transition
matrix of movers' $z$, denoted $G_{\text{move}}$ in the text.\label{fig:Contours-of-copula}}
\end{figure}


\section{Structural Model Fit\label{sec:Structural-Model-Fit}}

The fit is displayed in tables \ref{tab:fit-model-and-data} and \ref{tab:fit-model-and-data-flows}.
The upper panel of \ref{tab:fit-model-and-data} shows moments related
to mobility, the lower panel shows moments related to homeownership.
Regarding mobility, the fit is very good overall. The estimates for
the auxiliary model defined in \eqref{M-eq:auxmod-move} representing
the age profile in ownership also provide a good fit to the data.
Looking at table \ref{tab:fit-model-and-data-flows} we see that the
average flows into each region are very close to the data. 

Moving on to moments related to ownership, we see that the unconditional
mean of ownership is identical to the data moment. Conditioning by
region provides a more varied picture, with some regions overestimated
and others underestimated. The reason for this is that there is heterogeneity
in ownership rates by region which is not easily accounted for by
the fundamentals of regional house price and mean income alone.\footnote{There is large degree of house price heterogeneity at the local level
with is not in the model but which contributes to the average ownership
rate at the regional level. Local building regulations, rent control
or certain topographical features all influence the actual house price
that the local level; The price index used in the model incurs some
unavoidable aggregation error in this respect, and the same holds
for my estimate of the average rent to price ratio.} Remember that by taking prices and incomes as given, the model is
restricted to only few levers that affect the homeownership rate.
The main parameters in this respect are the utility premia $\xi_{1},\xi_{2}$
and the weight in the final period utility $\omega$. The model at
the moment overpredicts ownership in later periods of life. This is
visible from the intercept of the auxiliary model \eqref{M-eq:auxmod-h},
which relates the ownership rate to an age profile. The reason for
this is that in a model where age and wealth are the main dimensions
of variation across households, as soon as a certain wealth threshold
is crossed, all agents become owners. In other words, the model cannot
account for wealthy houeholds who prefer not to own.\footnote{One way to improve in this dimension would be to introduce different
types of housing preferences.}

Given that the CRRA coefficient $\gamma$ is taken as fixed in the
current implementation of the model, the moments relating to wealth
resulting from the model can be viewed as some form of model validation.
The model moments in table \ref{tab:fit-model-nontarget} are not
included in the SMM objective function, that is, they are not targeted
by the estimation algorithm. The model overpredicts total wealth accumulation,
related to the above mentioned slight overprediction of owners at
old age.

\begin{table}[p]
\begin{centering}
\input{\Dropbox/model/fit/moms.tex}
\par\end{centering}
\caption{Empirical targets and corresponding model moments. The auxiliary models
reference equations in the main text.\label{tab:fit-model-and-data}}
\end{table}

\begin{table}[p]
\begin{centering}
\input{\Dropbox/model/fit/moms-flows.tex}
\par\end{centering}
\caption{Empirical targets and corresponding model moments for population flows.\label{tab:fit-model-and-data-flows}}
\end{table}

\begin{table}[p]
\begin{centering}
\input{\Dropbox/model/fit/moms-nontarget.tex}
\par\end{centering}
\caption{Non-targeted model and data moments. This set of moments does not
enter the SMM objective function and can thus be seen as a form of
external validation of the model.\label{tab:fit-model-nontarget}}
\end{table}

Auxiliary models and out of sample predition

\begin{figure}
\begin{adjustbox}{center}

\includegraphics[scale=0.7]{../../../Dropbox/research/mobility/output/model/fit/fit_auxmods2}\includegraphics[scale=0.7]{../../../Dropbox/research/mobility/output/model/fit/fit_wealth}

\end{adjustbox}

\caption{Left panel: Parameters of the auxiliary models and table with resulting
implications for the model generated ownership rate (inset). Right
panel: out of sample predictions about average wealth conditional
on age and region. Wealth moments are not included in the SMM objective
function.\label{fig:fit-2}}
\end{figure}


\section{Additional Results}

\subsection{Elasticity of Migration wrt positive price shock}

The overall population elasticity is on average $-0.1$. Inflow elasticities
are unambigously negative for both incoming buyers and renters: both
find the region more expensive, hence stay away. Regarding outflows,
the picture is more nuanced. Notice that owners experience a positive
wealth shock in this case, which may (or may not) tip the balance
towards moving to another region, when previously this was suboptimal.
On aggregate, a one percent price increase leads to 1.1\% increase
in renter outflows, much larger than the corresponding 0.4\% increase
in owner outflows.

\begin{table}[H]
\begin{centering}
\input{\Dropbox/model/fit/elasticities_p.tex}
\par\end{centering}
\caption{Elasticities with respect to an unexpected and permanently positive
price shock by region. Statistics are computed identically as in table
\ref{tab:elasticities}. \label{tab:elasticities-p}}
\end{table}


\subsection{Comparative Statics of a Regional Price Shock}

The aim of this section is to illustrate how the model reacts to regional
price shocks in a comparative statics sense. This means that I will
shock one region at a time with a regional house price and income
shock, which deviates the observed price and income series to an unexpectedly
lower level in the year 2000. All other regions are kept constant
at baseline, observed, prices. The purpose of this exercise is to
show how regions differ in response to a given shock. It is important
to understand that the same sized shock can have very different results
in different regions.

The exercise is set up in partial equilibrium, as is indeed the model.
In the present context where we are interested in a ceteris paribus
effect of shocking one region only at each time, this seems to be
only a small limitation. We proceed thus in the folling fashion: Every
region is taken through different combinations of counterfactual regional
price and income shocks. For each region $d$, both $p_{dt}$ and
$q_{dt}$ may deviate in the year $t=2000$ by $\pm5\%$ with respect
to their observed (and expected) level, by surprise, and proceed at
this deviated level for ever after. The results from this are collected
in tables \ref{tab:shocks-by-region_95_95} through \ref{tab:shocks-by-region_105_105}.
Figure \ref{fig:Comparative-Statics} provides an illustration of
a prototypical regional shock.

\begin{figure}
\begin{centering}
\input{\Dropbox/model/data_repo/out_data_jl/shockregPaths.tex}
\par\end{centering}
\caption{Comparative Statics of Regional Shock. Dashed line is the shocked
series for a given region. This picture applies a 10\% shock to $Q$
and a 6\% shock to $P$. Both $y$-axis are in thousands of dollars.
\label{fig:Comparative-Statics}}
\end{figure}

\begin{table}[p]
\input{\Dropbox/model/experiments/shockRegions_ps_0.95_qs_0.95.tex}

\caption{shocks by region\label{tab:shocks-by-region_95_95}}
\end{table}

\begin{table}[p]
\input{\Dropbox/model/experiments/shockRegions_ps_1.0_qs_0.95.tex}

\caption{shocks by region\label{tab:shocks-by-region_1_95}}
\end{table}

\begin{table}[p]
\input{\Dropbox/model/experiments/shockRegions_ps_1.05_qs_0.95.tex}

\caption{shocks by region\label{tab:shocks-by-region_105_95}}
\end{table}

\begin{table}[p]
\input{\Dropbox/model/experiments/shockRegions_ps_0.95_qs_1.0.tex}

\caption{shocks by region\label{tab:shocks-by-region_95_1}}
\end{table}

\begin{table}[p]
\input{\Dropbox/model/experiments/shockRegions_ps_1.0_qs_1.0.tex}

\caption{shocks by region\label{tab:shocks-by-region_1_1}}
\end{table}

\begin{table}[p]
\input{\Dropbox/model/experiments/shockRegions_ps_1.05_qs_1.0.tex}

\caption{shocks by region\label{tab:shocks-by-region_105_1}}
\end{table}

\begin{table}[p]
\input{\Dropbox/model/experiments/shockRegions_ps_0.95_qs_1.05.tex}

\caption{shocks by region\label{tab:shocks-by-region_95_105}}
\end{table}

\begin{table}[p]
\input{\Dropbox/model/experiments/shockRegions_ps_1.0_qs_1.05.tex}

\caption{shocks by region\label{tab:shocks-by-region_1_105}}
\end{table}

\begin{table}[p]
\input{\Dropbox/model/experiments/shockRegions_ps_1.05_qs_1.05.tex}

\caption{shocks by region\label{tab:shocks-by-region_105_105}}
\end{table}


\subsection{Migration Shutdown with Changing Prices\label{subsec:Migration-Shutdown}}

This section presents the results from the experiment in section \ref{M-subsec:The-value-of}
in the main text under scenarios 2 and 3:
\begin{enumerate}
\item Baseline $\{q_{dt},p_{dt}\}_{t=1997}^{2012}$: Loss of migrants has
negligible impact on regional prices.
\item 1\% shock to $\{q_{dt},p_{dt}\}_{t=1997}^{2012}$: Local productivity
suffers a small loss.
\item 5\%/10\% shock: Large productivity decline and amplified effect on
house prices.
\end{enumerate}
Starting in table \ref{tab:shutdown-cons-small} with scenario 2,
we see the general pattern from the baseline experiment without changing
prices going through: Individuals dislike the counterfactual world,
with strong differences across regions and betwen age groups, and
between renters and owners at young age. With the 1\% shock on regional
income and house price, the compensation demanded is slightly higher
everywhere as compared to the baseline in table \ref{M-tab:shutdown-cons-base}.

Table \ref{tab:shutdown-cons-big} presents the corresponding results
for scenario 3, where the trend from scenario 2 continues: We see
the same pattern, just larger numbers.

\begin{table}
\begin{centering}
\input{\Dropbox/model/experiments/noMove_region_ctax_small.tex}
\par\end{centering}
\caption{Consumption compensation demanded after migration shutdown in scenario
2, i.e. regional prices decrease both by 1\% as a result of the shutdown
of migration.\label{tab:shutdown-cons-small} See table \ref{M-tab:shutdown-cons-base}
in the main text for the baseline experiment.}
\end{table}
\begin{table}
\begin{centering}
\input{\Dropbox/model/experiments/noMove_region_ctax_big.tex}
\par\end{centering}
\caption{Consumption compensation demanded after migration shutdown in scenario
3, i.e. regional prices and incomes decrease by 10\% and 5\% respectively
as a result of the shutdown of migration.\label{tab:shutdown-cons-big}
See table \ref{M-tab:shutdown-cons-base} in the main text for the
baseline experiment.}
\end{table}


\section{Welfare Measure\label{sec:Welfare-Measure}}

Denoting the lifetime utility from the baseline and policy regimes
under consumption tax $\Delta c$ by $V$ and $\hat{V}(\Delta c)$
respectively, the equalizing consumption tax $\Delta c^{*}$ solves
\begin{eqnarray*}
V-\hat{V}(\Delta c) & = & 0\\
V & = & \frac{1}{JN}\sum_{i=1}^{N}\sum_{t=1}^{J}\max_{k\in D}\left\{ v\left(x_{it},k\right)+\varepsilon_{ikt}\right\} \\
 & = & \frac{1}{JN}\sum_{i=1}^{N}\sum_{t=1}^{J}u(c_{it}^{*},h_{it}^{*},k_{it}^{*};x_{it})+\beta\mathbb{E}_{z,s,\mathbf{F}}\left[\overline{v}\left(x_{it+1}\right)|z_{ij},s_{ij},\mathbf{F}_{t}\right]\\
\hat{V}(\left(\Delta c\right)) & = & \frac{1}{JN}\sum_{i=1}^{N}\sum_{t=1}^{J}u(\left(\Delta c\right)\hat{c}_{it},\hat{h}_{it},\hat{k}_{it};\hat{x}_{it})+\beta\mathbb{E}_{z,s,\mathbf{F}}\left[\overline{v}\left(\hat{x}_{it+1}\right)|z_{ij},s_{ij},\mathbf{F}_{t}\right]
\end{eqnarray*}

where $N$ is the number of simulated individuals and $y^{*}$ indicates
the optimal choice of variable $y$. In other words, the welfare measure
is the average of over realized value functions \eqref{eq:V} in a
given simulation. Notice that the policy functions and resulting lifecycle
profiles $\hat{x}_{it}$ are different under the policy, for example
$\hat{c}\neq c$. Then, a value $\left(\Delta c\right)^{*}>1$ implies
that agents would be indifferent between any proposed policy change
if consumption were scaled up in every period, i.e. they would demand
a subsidy. In the opposite case of $\left(\Delta c\right)^{*}<1$
they would be happy to give up a fixed proportion $\left(\Delta c\right)^{*}$
of period consumption if they were given the opportunity to participate
in the policy. 

\section{Initial Conditions and Cohort Setup}

The SIPP estimation sample runs from 1998 through 2012. The data moments
the model is supposed to replicate are weighted averages over this
period, where the weights are the SIPP sampling weights. When reconstructing
an artificial sample from the model simulation, care must be taken
to replicate the shocks experienced by each cohort in the data leading
up to the point where they are observed. 

The data is subset to the ages allowed for in the model, i.e. 20\textendash 50.
I compute data moments, for example the average homeownership rate
in region $d$, or the average total wealth of age group 40\textendash 45
in $d$, as averages over the entire sample period:
\begin{eqnarray*}
\text{mean\_own\_data}_{d} & = & \frac{1}{15}\sum_{t=1998}^{2012}\left(\frac{1}{N_{dt}}\sum_{i\in d,t}^{N_{dt}}\omega_{it}\mathbf{1}\left[h_{it}=1\right]\right)\\
\text{mean\_wealth\_data\_40\_45}_{d} & = & \frac{1}{15}\sum_{t=1998}^{2012}\left(\frac{1}{N_{dt,j\in[40,45]}}\sum_{i\in d,t,j\in[40,45]}^{N_{dt,j\in[40,45]}}\omega_{it}w_{ijt}\right)
\end{eqnarray*}
where $N_{dt}$ is the number of people in $d$ at date $t$, and
$\omega_{it}$ is a person's crossectional weight, and $i\in d,t$
stands for $i$ is in $d$ at date $t$. Similarly, $i\in d,t,j\in[40,45]$
stands for $i$ is in $d$ at date $t$ and age $j$ in {[}40,45{]}.

This means that for the second data moment, for example, 40 year-olds
from 1998 contributed as well as 40 year-olds from the 2012 cohort.
Needless to say, those cohorts faced a different sequence of house
price shocks leading up the point of observation. For individuals
``born'' before the first data period, i.e. 1998, I construct regional
house price and regional income series going back until 1968. Simulating
individuals from the 1968 cohort for a full lifetime of J=30 years
until the reach age 50 brings them into the year 1998, where they
form the group of 50 year-olds in that particular year. This sort
of staggered simulation is carried out until the final cohort is born
in 2012 at age 20. No simulation needs to take place for any individual
alive at years after 2012. 

\section{Census Divisions}

\begin{figure}[H]
\begin{centering}
\includegraphics[scale=0.6]{../../../Dropbox/research/mobility/papers/us_divisions}
\par\end{centering}
\caption{Census Division Map, taken from \protect\url{https://www.census.gov/geo/maps-data/maps/pdfs/reference/us_regdiv.pdf}.
The Divisions are from left to right Pacific, Moutain, West North
Central, West South Central, East North Central, East South Central,
New England, Middle Atlantic and South Atlantic.\label{fig:Census-Division-Map}}
\end{figure}

\begin{table}[H]
\begin{centering}
\input{\Dropbox/data/sipp/own_p2y-edit.tex}
\par\end{centering}
\caption{Census Division abbreviations and characteristics. Shows average ownership
rates over 1997\textendash 2011 and median price to income ratios
for the same period. The (unobserved) house price for renters is computed
assuming an implied user cost of owning of 5\%, i.e. $p_{rent}=\frac{rent}{0.05}$.
\label{tab:SIPP-divisions}}
\end{table}


\end{document}
